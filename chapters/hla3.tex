\section{Motivation}

\section{Guide to use HCPC}
The tool can be used in two ways. First, a developer new to the code-base can load the abstraction tree which start with the root node. In the right side panel, for each node the number of execution paths, a brief natural text summary, and few frequent execution patterns are presented. Therefore, the developer can start first by observing summary, patterns of the root node. Now, the child nodes of the root node can be expanded and  similarly explored by observing corresponding node summary and patterns. The developer can continue this way according to their need to get acquainted with the coda-base behavior and high-level concepts in the code-base.

Second, a new contributor to a open source project can utilize the tool to understand high-level concepts related to a specific method. Developers first start from looking to open issues of a repository to find something work on. The issues are natural text description which provides information regarding a bug or a feature enhancement request. Developers can identify few keywords and use our tool to find matching methods relevant to the keywords. Next, a specific method can be selected to highlight relevant nodes in the tree. The difference between the first approach here is developers will be able to browse the tree with focus to the selected method. The node titles relevant to selected method will be highlighted so that the developer can expand their child nodes. By this way, the developer can start from the high-level concept to low-level source code related concepts for a specific method. By iterating this process, the developer can grasp high-level domain knowledge (with comment summary and IR techniques on function names) alongside insight into program execution scenarios which decreases the overhead due to lack of domain knowledge in the code-base. 

\section{Graphical demo of the tool} 