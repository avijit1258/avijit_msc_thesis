\label{chapter:conclusion}
In this thesis, we have worked on grouping execution paths of a software project for helping developers comprehend the codebase faster and locate related concepts for their tasks in hand. We start with existing works on clustering static execution paths for presenting high-level features in a software project. However, we find some limitations and scope of improvement to make the abstract code summary more usable for the daily activities of software developers. First, we experimented with different information retrieval techniques to find out which techniques provide more helpful labels for abstraction nodes. We also proposed using the terms in method names instead of whole method name as input for IR techniques. We also conducted a human subject study to find out how developers rate different IR techniques and compared automatic naming with manual naming by developers. From the study, we found TFIDF with terms in method are better supported by the manual labeling compared to LDA, LSI techniques. Moreover, developers preferred the words variant than the method variant of the labeling technique. Second, we proposed to add additional information such as node summary, execution patterns for each abstraction node to make the abstract code summary tree more comprehensible. We conducted a case study with three different subject systems to find the potential of attaching the two new information for each abstraction node. We found that attaching node summary, and execution patterns can complement node labels for more detailed understanding in relation to source code. However, we observed that using an agglomerative cluster tree poses some difficulty to browse as it presents all the clustering step by step. In addition, we noticed that it is difficult to explore the tree when targeting some specific keywords. In our third study, we addressed the issue of abstraction tree being overwhelming to browse by simplifying the agglomerative cluster tree using cluster flattening technique. Additionally, we have added an abstraction node highlighting technique for browsing the tree targeting specific keywords or methods. To evaluate the usefulness of our technique, we developed a web application called the HCPC. We performed a human subject study with an industry project and their developers. From the study, we found that the HCPC tool can help developers get started with a project alongside finding relevant execution patterns for specific tasks in hand.

In future, we plan to adopt automatic method summarizing techniques~\cite{wan2018improvingCodeSummary, ahmad2020transformerCodeSummary, zhu2019automaticSummaryReview} since in industry settings not every method is properly documented. We also plan to incorporate the GitHub project import option for exploring in the HCPC tool. Moreover, we will add feature to export reports with our analysis result. We have a plan to conduct a wide-scale user study with popular Python open-source projects.
