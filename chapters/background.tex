\section{Concept Cluster Tree}
\subsection{Leaf node}
\subsection{Intermediate node}
\subsection{Execution path}
\subsection{Entry and Exit node}
\section{Information Retrieval Techniques}
\subsection{TFIDF}
TFIDF is weight based statistical information retrieval technique. It tries to find important terms to a specific document by analyzing collection of documents. TFIDF is popular for document classification, search engine ranking and text mining\footnote{https://en.wikipedia.org/wiki/Tf–idf}. TFIDF ranks terms by term frequency-inverse document frequency score. Term frequency is count of a term in a document. Term frequency is biased towards common terms which mostly irrelevant to the document. 

\begin{equation}
    tf (W_x, D_x) = f_{W_x,D_x}
    \label{eq:tf_background}
\end{equation}
\begin{equation}
    idf(W_x) = \log(\frac{n}{df(W_x)})+1
    \label{eq:idf_background}
\end{equation}
\begin{equation}
    tf-idf(W_x, D_x) = tf(W_x,D_x) * idf(W_x)
    \label{eq:TFIDF_background}
\end{equation}


Jones \cite{jones1972statistical} introduced inverse document frequency which penalties common terms by counting their occurrence across the corpus. Let, $D = \{D_1, D_2, ..., D_n\}$ is a collection of documents and $W = \{W_1, W_2, ....., W_n\}$ is unique terms in the collection of documents. Now, to calculate term frequency for term $W_x$ in document $D_x$, we have to count frequency of term $W_x$ in  document $D_x$ which is required to calculate term frequency according to equation \ref{eq:tf_background}. In addition, we have to count the number of documents has term $W_x$ which is used to calculate inverse document frequency using equation \ref{eq:idf_background}. In equation \ref{eq:idf_background}, $n$ is the number of documents in the corpus and $df(W_x)$ is the number of documents which contain term $W_x$. Equation \ref{eq:TFIDF_background}, multiplies term frequency and inverse document frequency to reward significant terms and penalize common terms. 


\subsection{LDA}
\subsection{LSI}
\section{Jaccard Distance}
Jaccard Distance can measure similarity between two sequences according to equation \ref{eq:jaccard}. For example, we have two execution path $E_i$ and $E_j$ and they have set of function names $F_i$ and $F_j$ respectively. Therefore, similarity between $E_i$ and $E_j$ can be measured by equation \ref{eq:jaccard}. 
\begin{equation}
\label{eq:jaccard_similar}
    JD\_similar(E_i, E_j) =  \frac{F_i \bigcap F_j}{F_i \bigcup F_j}
\end{equation}

\begin{equation}
\label{eq:jaccard_dissimilar}
    JD\_dissimilar(E_i, E_j) =  1 - \frac{F_i \bigcap F_j}{F_i \bigcup F_j}
\end{equation}
If $E_i$ and $E_j$ are very similar, according to equation \ref{eq:jaccard_similar} similarity score will be near 1 and vice-versa. Clustering algorithm merges those two clusters which distance measures are minimum. Equation \ref{eq:jaccard_dissimilar} subtract Jaccard Distance by 1 to get desire dissimilarity measure for clustering algorithms.

\section{Agglomerative Hierarchical Clustering}
Clustering algorithms are popular in many data mining, unsupervised machine learning and pattern recognition applications. Clustering algorithm try to group similar observations together to find significant patterns in the observations. Hierarchical clustering can be done in two ways. One is bottom-up (agglomerative) and another is top-down (devisive). For Devisive clustering, all observations starts in a single cluster and divided into different clusters using heuristics. Agglomerative clustering starts by considering observations as individual clusters and then group them until all observations end-up in the same cluster.

\emph{Include two image to clearly demonstrate how clustering works}

\section{Text Rank}
Mihalcea \cite{mihalcea2004textrank} proposed graph based ranking algorithm inspired by PageRank algorithm to rank entities in natural language. Two of the significant application of TextRank are keyword extraction and sentence extraction. Sentence extraction can be formulated to generate summary of natural language text. To generate summary of a  text, first, sentences are split as they are the unit for TextRank algorithm. Next, sentences are converted to word embedding vectors. In the next step, similarity matrix is computed from embedding vectors. Then, a graph is created where vertices are sentences and edges represent similarity scores between sentences\footnote{https://www.analyticsvidhya.com/blog/2018/11/introduction-text-summarization-textrank-python/}. Similarity scores are used to extract top ranked sentences according to equation \ref{eq:textrank}.

\begin{equation}
\label{eq:textrank}
    WS(V_i) = (1 - d) + d * \sum_{V_j\epsilon IN(V_i) } \frac{w_{ji}}{\sum_{V_k \epsilon Out(V_j)} w_{jk}}  WS(V_j)
\end{equation}

Let, $ G = (V, E)$ is a directed graph where V is the collection of vertices and E is the collection of edges. $In(V_i)$ is the set of vertices which points to vertex $V_i$. Similarly, $Out(V_j)$ is the set of vertices which vertex $V_j$ points to. The similarity score between vertex $V_i$ and $V_j$ is represented by $w_{ji}$. 
\section{Sequential Pattern Mining}
