In this chapter, we provide a brief description of the thesis. In Section \ref{intro:motivation}, we discuss motivation of the thesis. Then we addressed three problems in Section \ref{intro:problem}. In Section \ref{intro:research_questions} and \ref{intro:solution}, we have introduced the research questions and provided brief summary of our solutions. In Section \ref{intro:outline}, we outline the whole thesis chapters. 


\section{Motivation}
\label{intro:motivation}
  The growing demands of new requirements for software applications make the codebase large. As the life-cycle of software increases, more resources are devoted to the maintenance of the software. If developers want to add a new feature or fix bugs in the existing features, they need to understand related domain knowledge alongside relevant code structure. The ratio of reading code versus writing code in a software developer's role is over 10 to 1 \cite{martin2008clean}. In addition, if a new developer joins the team, they need to understand how the high-level feature maps with existing low-level source code. When a software developer has to implement a new feature or enhance an existing feature, they need to look for the relevant methods, classes and files to understand how different parts of the relevant code interacts. After getting a good grasp of the relevant codebase, the developer can start working on the new feature. The process of understanding source code is called program comprehension. However, depending on the knowledge a developer possesses for a specific codebase, the steps for comprehending the program can be different.
 
%  The process of looking for relevant codebase to fix a bug or implement a new feature is called concept location. Existing techniques for concept location focus on keyword based searching in the codebase. Code search engines return relevant files, classes when developers search for particular keywords. 
 
The program comprehension techniques mainly consist of two models ~\cite{tilley1998reverseEngineeringFramework, von1993programToolRequirements, siegmund2016programPastFuture}. One is the top-down model, where developers have the system's domain knowledge and tries to map bottom-level source code to the high-level domain knowledge (features in a system). In many cases, the developers lack domain knowledge, forcing them to go through low-level codebase and gradually build high-level knowledge. The process of cognitive mapping from source code to domain knowledge is called the bottom-up model. When the codebase is new or unknown to the developers, and they lack domain knowledge, generally, the bottom-up model is followed by the developers~\cite{wei2002surveyCategorizationComprehension, siegmund2016programPastFuture}. The top-down model is more flexible and efficient than the bottom-up model for developers to have some idea about what to expect in the codebase or where to start from~\cite{brooks1983theoryComprehensionPrograms}. 
 
 As program comprehension is an integral part of software maintenance, effective tool support for program comprehension will help developers do their day-to-day job properly and with minimal cognitive load. The tool support for program comprehension can save valuable human resources, which cut the overall cost for software maintenance~\cite{kruger2019features}. Developers prefer to have high-level domain knowledge and then map the source code to the domain knowledge~\cite{brooks1983theoryComprehensionPrograms}. However, in real-world scenarios, developers in the industry and open source projects have to resolve issues with no option except to follow the cognitive heavy bottom-up model. For example, GitHub, home to many open source projects, has 56 million developers who have completed 1.9 billion contributions\footnote{https://octoverse.github.com/} in the range of October 2019 - September 2020. The tech giants companies like Google, Apple, Facebook, Microsoft have dedicated developer times for contributing to open source projects. Visual Studio Code, currently the most popular code editor from Microsoft, is developed by more than thousands of developers across the globe\footnote{https://github.com/microsoft/vscode}. The developers except the core team mainly fixes bugs or implements new features without being familiar with the whole codebase. In the first step of their contribution, they must acquaint themselves with relevant parts of the codebase, which is the bottom-up model. Therefore, it is essential to have sophisticated tools to help the developers with the bottom-up model. Researchers work on abstracting source code based on call graphs to reduce the cognitive load when developers follow the bottom-up model. 


Method names are the lowest level abstraction in the source code. Method names represent a unit task of the overall system \cite{de2012IRMethodsArtifacts, starke2009searching}. The interaction between different methods is the building block to understand the high-level concept in source code. Call graphs are visual representations of interactions among methods in the system. Call graphs construction techniques are two types. The static call graphs are built by analyzing source code to find the caller-callee relationships among methods. Later, building a graph using the relationships where edges represent which method calls which method and nodes represent the method names. The dynamic call graphs are constructed by logging function invocations during the run-time. To generate dynamic call graph, the software system needs to be run for different scenarios. During the scenario execution, function invocations are recorded, which can be converted to a graph similar to the static call graph. The main difference between dynamic call graph and static call graph is that the dynamic call graph contains the only methods invoked during the execution where the static call graph includes all the methods in the codebase ~\cite{gharibi2018automaticStaticCluster}. The advantage of a dynamic call graph is that the call graph can be generated for targeted execution scenarios ~\cite{feng2018hierarchicalExecutionComprehension}. One disadvantage of the dynamic call graph is that it generates a massive amount of redundant data (logged information of repeated function executions), which is difficult to process. We have decided to use static call graphs to create a tool for supporting program comprehension models. 

As the static call graph properties align more to build an abstract code summary, recently, few studies are utilizing static call graph to generate abstract code summary of a software system~\cite{gharibi2018automaticStaticCluster, walunj2019graphevoEvolutionCall}. In this thesis, we focus on enhancing the capability of an abstract code summary from the existing research by addressing limitations. We also focus on the usability of the abstract code summary tree by building an interactive program comprehension tool
in a guided way according to their specific tasks. 
 
\newpage


\section{Problem }
\label{intro:problem}
% In this section, we state three sub-problems addressed in this thesis.
    \subsection{Sub-Problem \#1: Lack of human evaluation and comparison between IR techniques} In the literature, a great many studies have focused on generating an abstract code summary of a software system using both static and dynamic call graph~\cite{feng2018hierarchicalExecutionComprehension, gharibi2018automaticStaticCluster, xin2019identifyingFeaturesExecution}. The abstract code summary is a tree-like structure where execution scenarios are clustered, and each node is labelled using different information retrieval (IR) techniques on source code entities. The success of constructing the abstraction tree depends on how well the labelling techniques perform. Other information retrieval techniques show promising performance in naming source code artifacts \cite{chen2016topicMiningRepositories, panichella2013topicModelsTasks, sun2016surveyTopicSE}. Although a lot of work exists on hierarchical abstraction, they lack comprehensive study on the effectiveness of different information retrieval techniques in labelling nodes of an abstraction tree with humans in the loop. No empirical research exists to find which IR technique works well in which situation.
Moreover, methods are treated as a unit \cite{gharibi2018automaticStaticCluster, feng2018hierarchicalExecutionComprehension} while using different information retrieval techniques for labelling nodes. Previous research \cite{de2012IRMethodsArtifacts} shows that IR techniques perform better when more information like comments are used instead of method names. Therefore, using method names as unit provides less opportunity to retrieve the overall context.
    

    \subsection{Sub-Problem \#2: Abstraction nodes are too short for helpful comprehension}
    In the previous studies~\cite{feng2018hierarchicalExecutionComprehension, gharibi2018automaticStaticCluster}, each node has five method names as their label in the abstract code summary tree. During our first study to evaluate IR techniques on labelling nodes, we observe that using 5-10 method names or words serves as a title for the node. The title can provide context, although it is difficult to comprehend what is happening inside a node without further detail. Each abstraction node is a collection of execution paths that may have variable lengths. Providing all the execution paths of a node to developers hinders the purpose of abstraction. Therefore, the challenge is to develop a solution that can briefly provide the context of a node without providing everything. 
    
   
    \subsection{Sub-Problem \#3: Making the abstraction tree usable for software engineering tasks} 
    
    Newcomers to open source software struggle with a lack of domain knowledge. Usually, developers (contributors) look for trending projects in their choice of language and popularity to contribute in social coding platforms (GitHub, Gitlab). As most of the time, the problem being solved is unknown to the developer, developers struggle to map low-level source code to high-level concepts. As it is stated in previous studies~\cite{brooks1983theoryComprehensionPrograms}, developers prefer the top-down model to browser source code for program comprehension. In the top-down model, developers have some domain knowledge, which they later try to map with source code. The hierarchical abstraction tree has the potential to bridge the gap between the top-down and bottom-up cognition models. However, the challenge is to tailor the abstraction tree for the developers to use for a specific task in hand or target a particular unit of the source code (method). 
    
    % Hierarchical abstraction tree help developers to understand high-level domain knowledge. When developers have domain knowledge, they tend to follow top-down model. As the abstraction tree captures high-level summary of whole software system, it is difficult to browse the abstraction tree for a specific task in hand. Browsing the abstraction tree by a guided way driven by task in hand can make the tree more usable in software maintenance. 
    
\section{Research Questions}   
\label{intro:research_questions}
While considering the above problems discussed in Section \ref{intro:problem}, we came up with below mentioned five research questions.
\begin{itemize}
    \item RQ1: How well the automatic labeling with the candidate approaches performs compared to manual labeling?
    \item RQ2: What are the developers' preferences over full method names and terms in method names as node title?
    
    \item RQ3: How can we provide a natural text summary to abstraction nodes?
    \item RQ4: How can we mine significant patterns from execution paths for each abstraction node?
    \item RQ5: How can we make the abstraction tree useful for daily day-to-day software engineering jobs?
    
\end{itemize}

Research question one and two correspond to \emph{sub-problem 1}, research question three and four correspond to \emph{sub-problem 2} and research question five correspond to \emph{sub-problem 3}. This research aims to help software development activities by generating abstract code summary using call graphs. 



\section{Solution}
\label{intro:solution}
Considering the three problems mentioned above statements (Section \ref{intro:problem}) in the domain of program comprehension, we contributed three studies. Below we have briefly discussed the three studies.  

\subsection{Labeling abstraction nodes and human evaluation}

In this study, by mining concepts from source code entities (names of functions/methods), we generate an abstract code summary tree
with improved naming of the cluster nodes. Our motivation is to complement existing studies to facilitate more effective program comprehension for
developers to address \emph{problem statement \#1}. We apply three different information retrieval techniques such as TFIDF (Term frequency-inverse document frequency)~\cite{ramos2003usingTfidfRelevance}, LDA (Latent  Dirichlet  Allocation)~\cite{blei2003latentLDA}, and LSI (Latent Semantic Indexing)~\cite{deerwester1990indexingLSI} (i.e., each technique with function
names and words in function names variation) to label nodes of an abstract code summary tree generated by clustering execution paths. Our experiment found that among the techniques on average, TFIDF performs better with around 64\% matching than the other
two methods (LDA and LSI) that show 37\% and 23\% matching respectively with names suggested by the users for 12 cases. Besides,
the words in a function name variant perform at least 5\% better in the user rating for all the three techniques on average for the use cases.
Our study draws on the existing research but considers more techniques and humans response for comprehending outputs using the three
techniques.

\subsection{Providing summary and significant patterns for abstraction nodes}

In this study, we develop two new techniques to supplement nodes information in hierarchical abstraction tree for better comprehension to address \emph{sub-problem \#2}. Generally, methods are expected to come with documentation at the start with a single line describing what the function does unless the method is concise and obvious \footnote{https://google.github.io/styleguide/pyguide.html}. First, we tried to exploit this standard practice for generating a brief text summary for each node. 
To complement existing techniques of labeling nodes, we add a text description to the node by summarizing all the method comments under that node. 

Second, execution paths in the call graph represent execution scenarios~\cite{salah2005scenariographer,pradel2009automatic}. Therefore, inspired by previous studies \cite{salah2005scenariographerReverseEngineering, pradel2009automaticUseageSpecification} we add significant patterns for each node by analyzing all execution paths under the node. We conduct an empirical study with three subject systems to evaluate the potential of the two proposed techniques. We found that the proposed techniques complement the existing abstraction tree, although there are some challenges. By addressing those challenges, the proposed techniques will be more effective for program comprehension. 


\subsection{Finding effectiveness of abstract code summary tree}
As discussed in the \emph{sub-problem #3}, making the abstraction tree browse-able with a specific target is helpful for new contributors in open source software systems. Having a system that helps top-down cognition possible without domain knowledge can be a game-changer for new contributors. In this study, we have built a system where the tree can be browsed by selecting a specific method. When a particular method is selected, relevant nodes in the tree are highlighted. Moreover, developers can see information like files involved, number of execution paths, summary and frequent patterns of an abstraction node. To evaluate the effectiveness, we have conducted a user study with the developers from the \textbf{Scidatamanager}\footnote{http://scidatamanager.usask.ca} team. The participants evaluated our approach on the abstract code summary tree generated from the Scidatamanager project. From the participant's feedback, it is viable that the HCPC tool can help developers get an overview of a codebase. In addition, the HCPC tool is helpful to know the relevant method, files to be looked for doing a particular task. 

% In this study, we built an interactive system that helps developers use the abstraction tree in a guided way to address \emph{problem statement \#3}. When Developers search with keywords in the tree, the system highlight the relevant nodes. So, the developers can browse the abstraction tree for specific task in hand. We conducted a user-study to evaluate the effectiveness of the system. 

\section{Publications}

Below we have listed published and submitted works (with collaborators) from this thesis. 

\begin{itemize}
    \item \emph{Avijit Bhattacharjee}, Banani Roy and Kevin Schneider. Clustering execution scenarios to aid top-down
model of program comprehension. 37th International Conference on Software Maintenance and Evolution. (Submitted)
    \item \emph{Avijit Bhattacharjee}, Banani Roy and Kevin Schneider. Human centric program comprehension bygrouping static execution scenarios.36th IEEE/ACM International Conference on Automated Software Engineering (ASE). (To be submitted)
    \item \emph{A. Bhattacharjee}, S. Nath, S. Zhou, D. Chakroborti, B. Roy, C. Roy, and K. Schneider. An Exploratory Study to Find Motives behind Cross-platform Forks from Software Heritage Dataset. In Proceedings of the 17th International Conference on Mining Software Repositories (MSR) - Mining Challenge Track, 2020.
    \item Saikat Mondal, C M Khaled Saifullah, \emph{Avijit Bhattacharjee}, Mohammad Masudur Rahman, and Chanchal K. Roy. 2021. Early Detection and Guidelinesto Improve Unanswered Questions on Stack Overflow. InProceedings of14th Innovations in Software Engineering Conference (formerly known as In-dia Software Engineering Conference), Bhubaneswar, Odisha, India, February25–27, 2021 (ISEC 2021),11 pages
    
    
\end{itemize}

\section{Outline of the Thesis}
\label{intro:outline}

In Chapter \ref{chapter:background}, we discuss some background on the call graph-related terminologies, clustering technique, different information retrieval techniques alongside a text summary technique. Chapter \ref{chapter:hla1} focuses on different information retrieval techniques with human evaluation. In Chapter \ref{chapter:hla2}, we proposed two techniques for adding node summary and execution patterns in the abstraction tree to aid developers program comprehension. Finally, in Chapter \ref{chapter:conclusion} we conclude the overall summary of the thesis and discuss some future plan. 