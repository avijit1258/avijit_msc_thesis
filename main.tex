\documentclass{uofsthesis-cs}

% \usepackage{algorithm}
\usepackage{tikz}
\usetikzlibrary{arrows,snakes,backgrounds}
\usepackage{pgfplots}
\usepackage{pgfplotstable}
\usetikzlibrary{patterns}
\usepackage{hyperref}
\usepackage{amsmath}
\usepackage{todonotes}
\usepackage{subcaption}
\usepackage{graphicx}
\usepackage{listings}

\usepackage[linesnumbered,ruled]{algorithm2e}

% Documentation for the uofsthesis-cs class is given in uofsthesis-cs.dvi
% 
% It is recommended that you read the CGSR thesis preparation
% guidelines before proceeding.
% They can be found at http://www.usask.ca/cgsr/thesis/index.htm

%%%%%%%%%%%%%%%%%%%%%%%%%%%%%%%%%%%%%%%%%%%%%%%%%%%%%%%%%%%%%%%%%%%%%%%%%%%%%%
% FRONTMATTER - In this section, specify information to be used to
% typeset the thesis frontmatter.
%%%%%%%%%%%%%%%%%%%%%%%%%%%%%%%%%%%%%%%%%%%%%%%%%%%%%%%%%%%%%%%%%%%%%%%%%%%%%%

% THESIS TITLE
% Specify the title. Set the capitalization how you want it.
\title{HCPC: Human centric program comprehension by grouping static execution scenarios}

% AUTHOR'S NAME
% Your name goes here.
\author{Avijit Bhattacharjee}

% DEGREE SOUGHT.  
% Use \MSc or \PhD here
\degree{\MSc}         

% THESIS DEFENCE DATE
% Should be month/year, e.g. July 2004
\defencedate{July/2021}


% NAME OF ACADEMIC UNIT
%
% The following two commands allow you to specify the academic unit you belong to.
% This will appear on the title page as
% ``<academic unit> of <department>''.
% So if you are in the division of biomedical engineering you would need to do:
% \department{Biomedical Engineering}
% \academicunit{Division}
%
% The default is ``Department of Computer Science'' if these commands
% are not given.
%
% If you are in a discipline other than Computer Science, uncomment the following line and
% specify your discipline/department.  Default is 'Computer Science'.
% \department{If not Computer Science, put the name of your department here}

% If you are not in a department, but say, a division, uncomment the following line.
% \academicunit{Put the type of academic unit you belong to here, e.g. Division, College}


% PERMISSION TO USE ADDRESS
%
% If you are not in Comptuer Science you will want to change the
% address on the Permission to Use page.  This is done using the
% \ptuaddress{}.  Example:
%
% \ptuaddress{Head of the Department of Computer Science\\
% 176 Thorvaldson Building\\
% 110 Science Place\\
% University of Saskatchewan\\
% Saskatoon, Saskatchewan\\
% Canada\\
% S7N 5C9
% }

% ABSTRACT
\abstract{
New members of a software team can struggle to locate use requirements if proper software engineering principles are not practiced. Reading through code, finding relevant methods, classes and files take a significant portion of software development time. Many times developers have to fix issues in code written by others. Having good tool support for this code browsing activity can reduce human effort and increase overall developer productivity. To help program comprehension activities,  building an abstract code summary of a software system from the call graph is an active research area. A call graph is a visual representation of caller-callee relationships between different methods of a software project. Call graphs can be difficult to comprehend for a larger code-base. The motivation is to extract the essence from the call graph by finding execution scenarios from a call graph and then cluster them together by concentrating the information in the code-base. Later, different techniques are applied to label nodes in the abstract code summary tree. In this thesis, we focus on static call graphs for creating an abstract code summary tree as it clusters all possible program scenarios and groups similar scenarios together. Previous work on static call graph clusters execution paths and uses only one information retrieval technique without any feedback from developers. First, to advance existing work, we introduced new information retrieval techniques alongside human-involved evaluation. We found that developers prefer node labels generated by terms in method names with TFIDF (term frequency-inverse document frequency). Second, from our observation, we introduced two types of information (text description using comments and execution patterns) to abstraction nodes for better insight on what execution scenarios they cover. Finally, we introduced an interactive software tool which can be used to browse the code-base in a guided way by targeting specific units of the source code. In the user study, we found developers can use our tool to overview a project alongside finding help for doing particular jobs such as locating relevant files and understanding relevant domain knowledge.
}

% THESIS ACKNOWLEDGEMENTS -- This can be free-form.
\acknowledgements{
First, I would like to express my gratitude to my supervisor Dr. Banani Roy for her constant guidance, suggestions, motivation and patience during my thesis work. I am grateful to Dr. Kevin A. Schneider for his support, feedback and guidance to shape my thesis work at the end of my program when Dr. Banani was on leave for family emergency. I am grateful to Dr. Chanchal K. Roy for his support, feedback during my program.


I would like to thank Dr. Gordon McCalla, Dr. Shahedul Khan, and Dr. Madison Klarkowski for their willingness to take part in the evaluation and advisement of my thesis. In addition, I am grateful to them for their valuable feedback, and suggestions.  

I would like to thank anonymous reviewers for their valuable comments and feedback which helped to improve this thesis work.

Special thanks goes to the Software Research Lab (SRLab) and Interactive Software Engineering (iSE) lab members for the good time we have together during daily discussion at Tim's, playing cricket, soccer during summer. In particular, I would like to thank Dr. Manishankar Mondal, Dr. Masudur Rahman, Amit Kumar Mondal, Daniel Abediny, Muhammad Mainul Hossain, CM Khaled Saifullah, Shamse Tasnim Cynthia, Zonayed Ahmed, Naz Zarren Oishie, Shamima Yeasmin, Farouq Al-omari, Sristy Sumana Nath, Golam Mostaeen, Kawser Wazed Nafi, Debasish Chakroborti, Saikat Mondal, Md Nadim, Md Shamimur Rahman, Judith Islam, Tonny Kar, Md. Abdul Awal, Hamid Khodabandehloo.

I would like to thank the Computer Science department of the University of Saskatchewan for their financial assistance through scholarships, awards, bursaries which helped me to focus on this thesis work. Moreover, I would like to thank all the staffs of the Department for their constant support. In particular, I would like to thank Sophie Findlay, Heather Webb, Greg Oster, Shakiba Jalal, James Ko, Jeff Long, Maurine Powell, and Cary Bernath.

I would like to thank my friends and family who instead of being thousand miles apart was available for me when I needed them. In addition, I would like to thank Afsana Sultana Ruma, Kinkshuk Kalyan Sarker, Tanushree Das and Ushasi Srija Chakroborti for their time, love and mental support during my stay in Saskatoon. 
}

% THESIS DEDICATION -- Also free-form.  If you don't want a dedication, comment out the following
% line.
\dedication{

I dedicate this thesis to my mother Bina Bhattacharjee, my father Arun Kumar Bhattacharjee and my younger brother Bishwajit Bhattacharjee, who always believed in me and inspired me to become the best version of myself.
}

% LIST OF ABBREVIATIONS - Sample  
% If you don't want a list of abbreviations, comment the following 4 lines.
\loa{\abbrev{ACST}{Abstract code summary tree}
\abbrev{EP}{Execution Path}
\abbrev{IR}{Information Retrieval}
}

%%%%%%%%%%%%%%%%%%%%%%%%%%%%%%%%%%%%%%%%%%%%%%%%%%%%%%%%%%%%%%%%
% END OF FRONTMATTER SECTION
%%%%%%%%%%%%%%%%%%%%%%%%%%%%%%%%%%%%%%%%%%%%%%%%%%%%%%%%%%%%%%%%

\begin{document}

% Typeset the title page
\maketitle

% Typeset the frontmatter.  
\frontmatter

%%%%%%%%%%%%%%%%%%%%%%%%%%%%%%%%%%%%%%%%%%%%%%%%%%%%%%%%%%%%%%%%
% FIRST CHAPTER OF THESIS BEGINS HERE
%%%%%%%%%%%%%%%%%%%%%%%%%%%%%%%%%%%%%%%%%%%%%%%%%%%%%%%%%%%%%%%%

\chapter{Introduction}

\section{Motivation}
 The growing complexity of software applications requires large codebase. As the life-cycle of a software increases, more resources are devoted to the maintenance of the software. If some developer wants to add a new feature or fix bugs in existing features, they need to understand related domain knowledge alongside relevant code structure. The ratio of reading code versus writing code in a software developers role is over 10 to 1 \cite{martin2008clean}. In addition, if a new developer joins the team, they need to understand how the high-level feature maps with existing low-level source code. When a software developer has to implement a new feature or enhance an existing feature, they need to look for the relevant files, classes to understand how different part of the relevant code interacts. After getting a good grasp on the relevant codebase, the developer can start working on the new feature. The process of looking for relevant codebase to fix a bug or implement a new feature is called concept location. Existing techniques for concept location focus on keyword based searching in the codebase. Code search engines return relevant files, classes when developers search for particular keywords. 
 
 However, the program comprehension techniques mainly consist of two models. One is the top-down model where developers have the domain knowledge of the system and tries to map bottom-level source code to the high-level domain knowledge. In many cases, the developers lack the domain knowledge which force them to go through low-level codebase and gradually build the high-level knowledge. The process of cognitive mapping from source code to domain knowledge is called bottom-up model. When the codebase is new or unknown to the developer and developers lack domain knowledge, generally bottom-up model is followed by the developers. The top-down model is more flexible and efficient as developers have some idea what to expect in the codebase or where to start from. 
 
 As program comprehension is an integral part for software maintenance, better tool support for program comprehension will help developers do their day-to-day job faster. Better tool support for program comprehension can save valuable human resource which cut the overall cost of a software maintenance. Developers prefer to have high-level domain knowledge and then map the source code to the domain knowledge. To reduce the dependence on bottom-up model, researchers work on building hierarchical abstraction which simulate the process of bottom-up model. Finally providing a tree like structure where high-level concepts in the software systems are presented by grouping topics in low-level source code. 
\newpage
In the source code, method names are the lowest level abstraction. The method names represent a single unit task of the overall system. The interaction between different methods are the building blocks to understand high-level concept in source code. Call graphs are visual representation of interaction between methods in the system. Call graphs construction methods are two types. The static call graphs are built by analyzing source code to find the caller-callee relationships between different methods. Later, building a graph using the relationships where edges represent which method calls which method and nodes represent the method names. Another type of call graph is dynamic call graph. The dynamic call graphs are constructed by logging function invocation sequences during the run-time. To generate dynamic call graph, the software system needs to be run for different scenarios. During the scenario execution, function invocations are recorded which can be converted to a graph similar to the static call graph. The main difference between dynamic call graph and static call graph is that the dynamic call graph contains the only methods invoked during the execution where the static call graph contains all the methods in the codebase. The advantage of dynamic call graph is that the call graph can be generated for a targeted execution scenarios. However, as we want to create a tool for supporting program comprehension models, including all possible scenarios is very crucial. One more disadvantages of dynamic call graph is that it generates huge redundant data which is difficult to process. 

As the static call graph properties align more with the goal of building hierarchical abstraction of a software system, recently few research is going on using static call graph to generate abstraction of a software system. In this thesis, we focus on improving the existing research findings with more sophisticated techniques. We also focus on the usability of the hierarchical abstraction by building an interactive system which helps developers to navigate the abstraction in a guided way according to their specific task. 

 
\newpage

\section{Problem Statement}

\begin{itemize}
    \item \textbf{Problem Statement \#1} In the literature, lot of study performed to generate hierarchical abstraction of a software system using both static and dynamic call graph. The hierarchical abstraction is a tree like structure where nodes are labeled using different techniques. The success of the abstraction tree depends on how well the labeling technique performs. Different information retrieval techniques show promise in naming source code artifacts. Although a lot of work exist on hierarchical abstraction, they lack comprehensive study on labeling techniques in labeling nodes in an abstraction tree. Moreover, methods are treated as term in the information retrieval techniques for labeling nodes. Using words in method signature as term for information retrieval techniques pose the potential for more sophisticated node labeling. 
    
    \item \textbf{Problem Statement \#2}
    Nodes in hierarchical abstraction tree have only few keywords extracted from method signatures. As the abstraction tree represents summary of the whole software system, labeling the nodes only with few keywords is not sufficient for a comprehensive understanding of the system. In the existing studies on hierarchical abstraction did not address this issue. Using method comment to generate a summary for the nodes  pave the way for a better comprehensive understanding. Execution paths in the call graph represent execution scenarios. As in the literature, execution paths are clustered for abstraction tree generation, providing significant patterns in the execution paths with the nodes further improves the comprehension. Overall, this two property addition in the abstraction tree improves the overall usefulness of the tree. 

    \item \textbf{Problem Statement \#3} 
    
    
\end{itemize}



\section{Our Contribution}
Considering the above problem statements in the domain of program comprehension through hierarchical abstraction, we contributed three studies. Below we have briefly discussed the three studies.  

\subsection{Study 1}

In this study, by mining concepts from source code entities (names of functions/methods), we generate a concept cluster tree
with improved naming of the cluster nodes to complement existing studies to facilitate more effective program comprehension for
developers to address \emph{problem statement \#1}. We apply three different information retrieval techniques such as TFIDF, LDA, and LSI (i.e., each technique with function
names and words in function names variation) to name nodes of concept cluster tree generated by clustering execution paths. Our experiment found that among the techniques on average, TFIDF performs better with around 64\% matching than the other
two methods (LDA and LSI) that report 37\% and 23\% matching respectively with names suggested by the users for 12 cases. Besides,
the words in function name variant perform at least 5\% better in user rating for all the three techniques on average for the use cases.
Our study draws on the existing research but considers more techniques and humans response for comprehending outputs using the three
techniques.

\subsection{Study 2}

In this study, we develop two new techniques to make nodes in hierarchical abstraction tree more understandable to address \emph{problem statement \#2}. First, to complement existing techniques of labeling nodes, we add a summary to the node by using all the method comments under that node. Second, inspired by previous studies we add significant patterns for each node by analyzing all execution paths under each node. We conduct an empirical study with three subject systems to evaluate the potential of the two proposed techniques. We found that proposed technique 

\subsection{Study 3}


% \section{Related Publications}

\section{Outline of the Thesis}
In Chapter \ref{chapter:background}, we discuss some background on call graph related terminology, clustering technique, different information retrieval techniques alongside one text summary technique. Chapter \ref{chapter:hla1} focuses on different information retrieval techniques with human evaluation. In Chapter \ref{chapter:hla2}, we proposed adding node summary and patterns in the abstraction tree to aid developers program comprehension. Finally, in Chapter \ref{chapter:conclusion} we conclude the overall summary of the thesis and discuss some future plan. 



% Since thesis chapters are very long and there are a lot of them, it is recommended
% that you put each chapter in a separate .tex file and \input each one of them
% in order.  For example:
%
% \input chap1.tex
% \input chap2.tex
% ...
%
% The \input command inserts contents of the specified file at the point of the command.

%%%%%%%%%%%%%%%%%%%%%%%%%%%%%%%%%%%%%%%%%%%%%%%%%%%%%%%%%%%%%%%
% SUBSEQUENT CHAPTERS (or \input's)  GO HERE
%%%%%%%%%%%%%%%%%%%%%%%%%%%%%%%%%%%%%%%%%%%%%%%%%%%%%%%%%%%%%%%

\chapter{Background}
% TO DO LIST Background
% LDA
% Strike a match
% Pyramid score
% Sequential pattern mining
% abstract code summary tree bring from later chapter and enhance
% Background related works for committee

\label{chapter:background}

In this chapter, we briefly discuss relevant terms, topics and techniques helpful to this thesis. In Section \ref{background:call_graph}, we elaborate terms relevant to a call graph. We then present an  abstract code summary tree in Section \ref{background:cct}. In Section \ref{background:motive}, we provide an abstract code summary tree for a sample calculator program using our system. In Section \ref{background:techniques}, we have elaborated different techniques and algorithms used in the thesis. In Section \ref{background:related_work}, we discuss related work for the studies done in the thesis.  

\section{Call graph}
\label{background:call_graph}
A \emph{call graph} is a control flow graph of a program showing calling relationships between functions. Each node of the graph represents a function and each edge $(a, b)$ represent calling relationship where function $a$ calls function $b$. Figure \ref{fig:bg_call_graph} shows a simple call graph with six nodes indicating functions and six edges indicating calling relationships. Call graphs can be of two types. One type is a static call graph. A static call graph contains all the possible program execution scenarios. To generate a static call graph, source code of the program is analyzed to find the relationships. A dynamic call graph represents one program run scenario. Therefore, a dynamic call graph is exact and limited to the scenarios used to generate the graph. To generate a dynamic call graph, logger or profiler is applied which generates call graph during run-time of the program.

\begin{figure}[h]	
	\centering
	\begin{subfigure}[h]{3in}
	\includegraphics[width=3in]{figures/background/call graph.png}
	\caption{A sample call graph}\label{fig:bg_call_graph}		
	\end{subfigure}
	\quad
	\begin{subfigure}[h]{3in}
		\includegraphics[width=3in]{figures/background/execution_paths.png}
		\caption{All execution paths from the call graph}\label{fig:bg_execution_path}
	\end{subfigure}
	\caption{Call graph with entry node, exit node and execution paths}\label{fig:1}
\end{figure}

An \emph{entry node} for a call graph is the node in which the number of incoming degrees is zero. In Figure \ref{fig:bg_call_graph}, the call graph has two entry nodes \emph{F0, F3}. No other nodes call the functions or nodes \emph{F0, F3}. That means program execution can start from these nodes.

An \emph{exit node} for a call graph is the node in which number of outgoing degrees is zero. In Figure \ref{fig:bg_call_graph}, the call graph has two exit nodes \emph{F0, F3}. The exit nodes \emph{F0, F3} do not call any other functions or nodes. That means program execution will end when we come to these nodes.


The \emph{execution paths} of a call graph are the all possible program execution scenarios. A program execution scenario consist of a function call sequence starting from a \emph{entry node} and ending to a \emph{exit node} of the call graph. In Figure \ref{fig:bg_execution_path}, all the execution paths from the call graph of Figure \ref{fig:bg_call_graph} are listed. The first node of the execution paths are the Entry nodes which is defined above. Similarly, the last node of the execution paths are the Exit nodes. 




\section{Abstract Code Summary Tree}
\label{background:cct}
In this thesis, we introduce a term called abstract code summary (ACS) tree. In an ACS tree each leaf node is attached to an execution path extracted from the call graph of a software system. The parent nodes of the leaf nodes are grouping of similar execution paths (leaf nodes). We call this intermediate nodes an abstraction node as it abstracts similar execution scenarios. Each abstraction node has three properties which are title, text summary and execution patterns. 

\begin{figure}[h]
  \centering
  \includegraphics[width=\columnwidth]{figures/background/cct.png}
  \caption{An abstract code summary tree with its different components }
  \label{background:cct}
\end{figure}
In Figure \ref{background:cct}, we present a ACS tree where \emph{4, 5, 6, 7} nodes are leaf node which are attached to execution paths. Nodes \emph{1, 2, 3} are abstraction nodes which are grouping of the leaf nodes. Each abstraction nodes has number of execution paths and we use different information from those execution paths to generate concepts for them. Node 3 has two execution paths which belong to node 6 and 7. Like all other abstraction nodes Node 3 will have title, text summary and execution patterns. The title of Node 3 will be generated using different information retrieval techniques utilizing method signatures. Next, the text summary of Node 3 will be generated by summarizing comments of the methods which belong to the execution paths of Node 3. The execution patterns for Node 3 will be generated by finding frequent patterns from the execution paths of Node 3.

\section{Motivational Example}
\label{background:motive}
To demonstrate how a software system's hierarchical abstraction will work, we have created a sample Calculator program. The program takes two numbers as inputs, validates the inputs, and prompts the user to input which operations they want to perform. Later, according to the input, addition, subtraction, multiplication, division can be performed. This is a brief functionality of the calculator program. We have provided the source code of the Calculator program in appendix \ref{appendix:calculator}.

In Figure \ref{fig:motivation}, we have presented the hierarchical abstraction of the Calculator program. From the figure, we can see our Calculator program has six execution paths. Their node numbers are from 0-5. 

\begin{figure}[h]
  \centering
  \includegraphics[width=\columnwidth]{figures/hla2/hla2_motivation.png}
  \caption{An abstract code summary of the calculator program (EP means Execution path or leaf node and AN means Abstraction Node)}~\label{fig:motivation}
\end{figure}

\textbf{Constructing the abstract code summary.} To generate the tree shown in Figure \ref{fig:motivation}, the following steps are followed.

\begin{enumerate}
    \item To get the caller-callee relationships from the source code of Calculator program, we use a static source code analyzer.
    \item We construct a static call graph from the extracted relationships of \emph{Calculator.py} program. 
    \item From the call graph, possible execution scenarios are generated  which are the execution paths shown in Figure \ref{fig:motivation} (EP 0 - 5).
    \item Similarity scores for each pair of execution paths are calculated which is used by the clustering algorithm to group the execution paths. As EP 0, 1, ...., 4 all have three common functions, the similarity measure between them  will be same.  
    \item A clustering algorithm starts grouping the execution paths by taking the most similar two first. In Figure \ref{fig:motivation}, we see that EP 0, 1 are grouped together as abstraction node (AN) 7.
    \item As AN 7 have EP 0, 1, we use information retrieval techniques on all the terms in functions names of EP 0, 1 to label the node AN 7. 
    \item Although keywords are helpful for providing hints to features, having a text description and frequent execution patterns for each abstraction node increases comprehension. In Table \ref{table:node_summary_patterns}, we presented node summary and execution patterns for AN 10, 11. 
    
\end{enumerate}



\begin{table*}[h]
\caption{Abstraction Nodes with summary and execution patterns}
    \centering
    \begin{tabular}{ |l | p{5cm} | p{10cm} | } 
 \hline
 AN & Node Summary & Execution Patterns \\ 
 \hline
 11 & This function multiplies two numbers. This function mod two numbers. This function subtract two numbers.  &  \bullet init \rightarrow two\_number\_input \rightarrow valid\_number. 
 \bullet init \rightarrow operations\_to\_do \rightarrow add\_two\_numbers. 
 \bullet init \rightarrow operations\_to\_do \rightarrow divide\_two\_numbers. 
\bullet init \rightarrow operations\_to\_do \rightarrow mod\_two\_numbers. 
 \bullet init \rightarrow operations\_to\_do \rightarrow multiply\_two\_numbers. 
 \bullet init \rightarrow operations\_to\_do \rightarrow subtract\_two\_numbers. \\ 
10 & This function mod two numbers. This function divide two numbers. This function subtract two numbers. & 
\bullet init \rightarrow operations\_to\_do \rightarrow add\_two\_numbers. 
\bullet init \rightarrow operations\_to\_do \rightarrow divide\_two\_numbers. 
\bullet init \rightarrow operations\_to\_do \rightarrow mod\_two\_numbers. 
\bullet init \rightarrow operations\_to\_do \rightarrow multiply\_two\_numbers. 
\bullet init \rightarrow operations\_to\_do \rightarrow subtract\_two\_numbers. 
\\ 
 \hline
\end{tabular}
    \label{table:node_summary_patterns}
\end{table*}



\textbf{Exploring the abstract code summary.}
\begin{itemize}
    \item Execution path 0 and 1 represent the functionality of multiplying two numbers and adding two numbers, respectively. For these two clusters, add and multiply are the two different jobs they are doing. Other functions of the two paths are similar. So, the abstraction of these two execution paths is abstraction node 7. Five keywords are picked as the abstraction of execution paths 0 and 1. From the keywords of node 7, it is clear that descendent nodes do addition and multiply on two numbers.
    \item Next, for node 10, we can see the keywords are add, divide, mod, multiply, and subtract. These five keywords indicate that the descendant nodes of 10 do these numerical operations. If we observe the five execution paths (EP 0 - 4), we find that they perform add, delete, mod, multiply operation on two input numbers. We can see that the five keywords of node 10 summarize the functionality of its descendants.
    \item Similarly, for node 11, the keywords are mod, multiply, subtract, valid, and number. We can see the right child node (node 5) of node 11 input two numbers and then validates it. Left descendants of node 11 perform numerical operations. So, the summary of node 11 contains three words relevant to operation and two for input validation.
\end{itemize}
   From our understanding, we can see that this is an almost human level summary for node 10. The summary presented in Figure \ref{fig:motivation} is generated using TFIDF scores on words in method names. 

\section{ Techniques and Algorithms}
\label{background:techniques}
In this Section, we discuss important techniques and algorithms used to construct ACS tree. In Subsection \ref{background:tfidf}, \ref{background:lda} and \ref{background:lsi}, we discuss TFIDF, LDA and LSI technique which are used for generating node title from method names. In Subsection \ref{background:JD}, we discuss Jaccard Distance which is used to calculate similarity between execution paths. In Subsection \ref{background:AHC}, we discuss AHC algorithm which is used to cluster execution paths. Finally, we discuss Text Rank algorithm which generate node summary from method comments in Subsection \ref{background:text_rank}.

\subsection{TFIDF}
\label{background:tfidf}
TFIDF~\cite{ramos2003usingTfidfRelevance} is a weight based statistical information retrieval technique. It tries to find important terms to a specific document by analyzing a collection of documents. TFIDF is popular for document classification, search engine ranking and text mining\footnote{https://en.wikipedia.org/wiki/Tf–idf}. TFIDF ranks terms by term frequency-inverse document frequency score. Term frequency is the count of a term in a document. Term frequency is biased towards frequent terms which mostly stop words and other fairly meaningless words irrelevant to the document. 

\begin{equation}
    tf (W_x, D_x) = f_{W_x,D_x}
    \label{eq:tf_background}
\end{equation}
\begin{equation}
    idf(W_x) = \log(\frac{n}{df(W_x)})+1
    \label{eq:idf_background}
\end{equation}
\begin{equation}
    tf-idf(W_x, D_x) = tf(W_x,D_x) * idf(W_x)
    \label{eq:TFIDF_background}
\end{equation}


Jones~\cite{jones1972statistical} introduced inverse document frequency which penalties common terms by counting their occurrence across the corpus. Let, $D = \{D_1, D_2, ..., D_n\}$ is a collection of documents and $W = \{W_1, W_2, ....., W_n\}$ is unique terms in the collection of documents. Now, to calculate term frequency for term $W_x$ in document $D_x$, we have to count frequency of term $W_x$ in  document $D_x$ which is required to calculate term frequency according to equation \ref{eq:tf_background}. In addition, we have to count the number of documents has term $W_x$ which is used to calculate inverse document frequency using equation \ref{eq:idf_background}. In equation \ref{eq:idf_background}, $n$ is the number of documents in the corpus and $df(W_x)$ is the number of documents which contain term $W_x$. Equation \ref{eq:TFIDF_background}, multiplies term frequency and inverse document frequency to reward significant terms and penalize common terms. We have adopted \texttt{TFIDFVectorizer} class of scikit-learn \cite{scikit-learn} library for implementing $TFIDF$ technique.

\subsection{LDA}
\label{background:lda}
Latent Dirichlet Allocation (LDA)~\cite{blei2003latentLDA} is a statistical model that tries to describe a set of documents by assuming they are created from some topics. LDA is a very popular topic modeling technique. LDA assumes every term in a document belongs to some topic. So, it assumes each term belongs to some topic and then performs analysis to find which assumptions are supported by statistics of the corpus. We have used Gensim \cite{gensim} library for implementing LDA for our approach.

\subsection{LSI}
\label{background:lsi}
Latent Semantic Indexing (LSI)~\cite{deerwester1990indexingLSI} focuses on information retrieval based on semantic similarity between words where the previous techniques focus on matching words in query with words of documents. The semantic concept used in LSI assumes semantically similar words appear together. Information retrieval techniques which matches words suffer two limitations. They are \emph{synonymy} and \emph{polysemy}. \emph{synonymy} is the issue where the same object is described by different words depending on needs, knowledge and linguistic habits. On the other hand, \emph{polysemy} refers to the fact that words have multiple distinct meanings in different contexts. LSI, first, starts with a  Term-Document matrix where all terms are presented in the rows and documents in the columns. Table \ref{tb:LSI_term_document} shows an example of a Term-Document matrix. 

\begin{table}[h]
    \centering
    \caption{Sample Term-Document matrix}
 \begin{tabular}{|c|c|c|c|c|c|c|}
    \hline
    
        & ship & boat & ocean & voyage & trip   \\
        \hline
        Document 1 & 1 & 0 & 1 & 0 & 0  \\
        Document 2 & 0 & 1 & 0 & 1 & 0    \\
        Document 3 & 1 & 0 & 0 & 1 & 1  \\
    \hline
    \end{tabular}
    
    \label{tb:LSI_term_document}
\end{table}


Single Value Decomposition (SVD) method is used to project the term-document matrix to reduced numnber of dimensions.
The reduced matrix by SVD is an approximation of the term-document matrix which is a representation of the semantic similarity between words in documents. If we need to find similarity between a query, the query is converted to similar representation and compared to find relevant documents. By using this technique, LSI can detect semantic similarity even when the terms are different. Similar to LDA, we used Gensim \cite{gensim} library for implementing LSI.   



\subsection{Jaccard Distance}
\label{background:JD}
Jaccard Distance can measure similarity between two sequences according to equation \ref{eq:jaccard}. For example, we have two execution path $E_i$ and $E_j$ and they have set of function names $F_i$ and $F_j$ respectively. Therefore, similarity between $E_i$ and $E_j$ can be measured by equation \ref{eq:jaccard}. 
\begin{equation}
\label{eq:jaccard_similar}
    JD\_similar(E_i, E_j) =  \frac{F_i \bigcap F_j}{F_i \bigcup F_j}
\end{equation}

\begin{equation}
\label{eq:jaccard_dissimilar}
    JD\_dissimilar(E_i, E_j) =  1 - \frac{F_i \bigcap F_j}{F_i \bigcup F_j}
\end{equation}
If $E_i$ and $E_j$ are very similar, according to equation \ref{eq:jaccard_similar} similarity score will be near 1 and vice-versa. Clustering algorithm merges those two clusters which distance measures are minimum. Equation \ref{eq:jaccard_dissimilar} subtract Jaccard Distance by 1 to get desire dissimilarity measure for clustering algorithms.

\subsection{Agglomerative Hierarchical Clustering}
\label{background:AHC}
Clustering algorithms are popular in many data mining, unsupervised machine learning and pattern recognition applications. Clustering algorithms try to group similar observations together to find significant patterns in the observations. Hierarchical clustering can be done in two ways. One is bottom-up (agglomerative) and another is top-down (divisive). For divisive clustering, all observations starts in a single cluster and divided into different clusters using heuristics. Agglomerative clustering starts by considering observations as individual clusters and then group them until all observations end-up in the same cluster.

\begin{figure*}[h]
  \centering
  \includegraphics[width=0.5\columnwidth, height=0.5\columnwidth]{figures/background/agglomerative_clustering.png}
  \caption{Agglomerative and Divisive clustering algorithm with a sample cluster forest}~\label{fig:agglomerative_clustering}
\end{figure*}
In Figure \ref{fig:agglomerative_clustering}, a visualization of how agglomerative and divisive clustering algorithm works are presented. Lets assume there are five observations \emph{a, b, c, d, e} and we have similarity score between all the pairs of the observations. First, we can see five observations are treated as five clusters. From the similarity score we found that clusters \emph{d and e} are most similar. Therefore, we group cluster \emph{d and e} together as a new cluster \emph{de}. Now, in the cluster forest we have four clusters. In the next step, cluster b and c are the most similar. So, agglomerative clustering algorithm will group cluster b and c as a new cluster \emph{bc}. The agglomerative clustering will continue to merge clusters together until there is only one cluster in the cluster forest. For this example, the final cluster \emph{(abcde)} consists of all the initial clusters. 

\subsection{Text Rank}
\label{background:text_rank}
Mihalcea \cite{mihalcea2004textrank} proposed a graph based ranking algorithm called TextRank inspired by the PageRank algorithm to rank entities in natural language. Two of the significant application of TextRank are keyword extraction and sentence extraction. Sentence extraction can be formulated to generate summary of natural language text. To generate a summary of a paragraph, first, sentences are split as they are the unit for TextRank algorithm. Next, sentences are converted to word embedding vectors. In the next step, similarity matrix is computed from embedding vectors. Then, a graph is created where vertices are sentences and edges represent similarity scores between sentences\footnote{https://www.analyticsvidhya.com/blog/2018/11/introduction-text-summarization-textrank-python/}. Similarity scores are used to extract top ranked sentences according to equation \ref{eq:textrank}.

\begin{equation}
\label{eq:textrank}
    WS(V_i) = (1 - d) + d * \sum_{V_j\epsilon IN(V_i) } \frac{w_{ji}}{\sum_{V_k \epsilon Out(V_j)} w_{jk}}  WS(V_j)
\end{equation}

Let, $ G = (V, E)$ is a directed graph where V is the collection of vertices and E is the collection of edges. $In(V_i)$ is the set of vertices which points to vertex $V_i$. Similarly, $Out(V_j)$ is the set of vertices which vertex $V_j$ points to. The similarity score between vertex $V_i$ and $V_j$ is represented by $w_{ji}$. 


\section{Related work}
\label{background:related_work}
\subsection{Program Comprehension in General}
\label{related:program_comprehension}
Program comprehension is a cognitive way of understanding software systems to perform different software maintenance tasks \cite{wei2002surveyCategorizationComprehension, siegmund2016programPastFuture}. Three different types of cognitive models \cite{tilley1998reverseEngineeringFramework, von1993programToolRequirements, siegmund2016programPastFuture} can be found in the literature which is followed consciously or unconsciously by developers. The comprehension models are Top-down, Bottom-up, and Integrated. When developers have prior domain knowledge about a software system, the top-down model is preferred as they can map domain knowledge to low-level source code hierarchically \cite{brooks1983theoryComprehensionPrograms}. On the other hand, when developers lack domain knowledge, they start with low-level source code and then group the functionality together to have a hierarchical abstraction of the system features \cite{shneiderman1979syntacticInteractionsModel, pennington1987stimulusMentalRepresentations}. Integrated model \cite{shaft1995relevanceDomainKnowledge, von1993programToolRequirements} is a mix of top-down and bottom-up approaches. The problem in hand and the target system have different properties in the real world, which demand switching between top-down and bottom-up models. Generally, a developer can have prior domain knowledge of a few portion and point-blank for the rest of the system. This situation deserves the adapted use of both top-down and bottom-up approaches.   


\subsection{IR Techniques to Name Source Code Artifacts}
\label{related:IR}
% \cite{mcburney2014improvingTopicSummarize}
% \cite{de2012IRMethodsArtifacts} \cite{panichella2013topicModelsTasks}
% \cite{chen2016topicMiningRepositories}
% Very very important \cite{sun2016surveyTopicSE}
As software repositories contain unstructured data, topic model techniques are widely applied for different software engineering tasks to retrieve information \cite{chen2016topicMiningRepositories, panichella2013topicModelsTasks, sun2016surveyTopicSE}. Most common tasks where topic models showed promising results are source code comprehension, feature location, refactoring, bug localization, and others \cite{sun2016surveyTopicSE}. Lucia et al. \cite{de2012IRMethodsArtifacts} conducted a study to see how information retrieval techniques perform compared to manual naming Java class files. Developers were asked to pick ten keywords for each class file, and top-10 words are picked using different topic model technique and custom heuristics. Their experiment shows that in 40\%-80\% cases, automatic and human labels overlap. 

\subsection{Reverse Engineering}
\label{related:reverse_engineering}
\subsubsection{Subsystem Identification}
Muller et al. \cite{muller1990composingSubsystemStructures} proposed subsystem detection algorithm using different clustering components like variable, procedure, and modules. 
According to Bass et al. \cite{bass2003softwareArchitecturePractice}, two types of software architecture are useful for understanding a complex software system. They are Conceptual and Concrete architecture. A conceptual architecture provides high-level abstraction skipping the code level details. On the other hand, concrete architecture shows the implementation level information. Roy et al. \cite{roy2008softwareArchitectureRecovery} propose and evaluate a framework for the incremental and iterative application of automated architecture recovery (using SWAG Kit) and architecture analysis (using SAAM.). They showed that the reverse engineering tool cannot recover a deeply understood conceptual architecture without SAAM's application but can create a reasonable basis towards that direction. Murphy et al.\cite{MurphyNotkin2001} show that by generating reflexion models from high-level model and source model (i.e., static call graphs), it is possible to facilitate program understanding to the novice developers. In this thesis, we try to automatically recover conceptual architecture from concrete architecture, reducing manual effort.

\subsubsection{Call Graphs to Abstract a Software System Behaviors}

Static and dynamic call graphs are used in literature to help developers comprehend a software system to aid different maintenance tasks \cite{feng2018hierarchicalExecutionComprehension, gharibi2018automaticStaticCluster, xin2019identifyingFeaturesExecution}. Feng et al. \cite{feng2018hierarchicalExecutionComprehension} proposed an approach to use dynamic call graphs for understanding a system's behavior. They instrumented the subject systems to generate execution traces of method entry and exit events. Later, they followed the duplication removal process and constructed a call graph from the execution traces. Execution phases from this dynamic call graph are clustered to get system behaviors. Similarly, Gharib et al. \cite{gharibi2018automaticStaticCluster}, and Vijay et al. \cite{walunj2019graphevoEvolutionCall} also adopted clustering of execution paths from call graphs of the static variant. Using a static call graph brings the benefit of capturing all possible scenarios and less redundant data to handle than dynamic call graph \cite{gharibi2018automaticStaticCluster}. 

\subsubsection{IR Techniques on the Hierarchical Abstraction of a Software System}
Feng et al. \cite{feng2018hierarchicalExecutionComprehension} proposed an approach to identify behaviors of a system by hierarchically abstracting dynamic call graph from execution traces. Sequential pattern mining is applied to mine significant portions from the execution phases. Hierarchical clustering is performed to group execution phases. Later, the clusters are labeled using the TFIDF score, where method signatures serve as terms and the phases as document. 
Paul et al. \cite{mcburney2014improvingTopicSummarize} used static call graph to hierarchically abstract a system. In their hierarchical view, each node represents a method. To mine the topics, keywords from methods are considered. Hierarchical Document Topic Model (HDTM) by \cite{weninger2012documentTopicHierarchies} Weninger et al. is adopted, which works on graph documents to mine topic. Gharib et al. \cite{gharibi2018automaticStaticCluster} took a different approach. They went further with the static call graph by extracting execution paths and then clustering the execution paths. Each cluster in the cluster tree is labeled using top-5 method names from Tfidf. Levy et al. \cite{levy2019understandingLargeHierarchical} found interviewing developers that two kinds of comprehension go for large scale hierarchical view. They are system comprehension and code comprehension. In this thesis, we tried to adopt static call graph analysis from Gharib et al. and then improve their labeling technique. Nodes of the cluster tree is considered as a behavioral abstraction unit of a system. Method comments are used to generate a description of the unit and sequential pattern mining to create sample examples. 


\subsection{How Developers Locate Features in Source Code}
\label{related:feature_locate}
Kruger et al.~\cite{kruger2019features} studied two data sets (67 developers IDE activity, 600 developers IR-based tool usage). They suggested that there is room for improvement in the existing code navigation, code search tools. The manual processes followed by developers to locate features are of mostly three types~\cite{damevski2016field, wang2011exploratory, revelle2005understanding}. First, developers use information retrieval based tools to query for feature related keywords. In this thesis, we have used IR based techniques to label nodes. Developers can use our tool to find keywords of their interest. Second, there is an  execution-based process where developers try to find execution scenarios where the feature is active. After finding relevant execution scenarios, developers debug the execution scenarios by setting breakpoints. In our second study, we have attached execution patterns to nodes which can be utilized by developers to know where to set the breakpoints for understanding a feature. Third, there is an exploration-based process where developers explore source code to understand method calls to find a feature. In the HCPC tool, we showed method execution patterns for each node. Our tool can also help developers in browsing code using an  exploration-based process.

\subsection{Program Comprehension with Static and Dynamic Call Graph }
Feng \textit{et al.} \cite{feng2018hierarchicalExecutionComprehension} proposed an approach to abstract execution traces for program comprehension. To get execution traces, they used BLINKY to instrument source code for getting method-invocation calls. Different test cases are used to generate execution traces for different scenarios. From dynamic logs, they have built phase trees that are created from caller-callee relationships of invoked methods. After deleting duplicate phases, they clustered unique phases using the Agglomerative hierarchical clustering algorithm. Next, they applied a mining technique to get frequent pattern phases of each level of clustered phase tree. For comprehension purposes, they used TFIDF to rank method names of frequent phases and then used the top 20 method names for the final label. Depending on dynamic call graphs comes with some limitations as it depends on the test cases heavily, and the size of log file generated is difficult to handle. Therefore, we choose static call graphs to remove the test dependency and capture a call graph's overall execution scenario. Gharib \textit{et al.}  \cite{gharibi2018automaticStaticCluster} proposed an approach using static call graphs for hierarchical abstraction. First, they generated a static call graph for a subject system that captures overall function relationships. Second, execution paths from the call graph are extracted, which become the building blocks for their approach. Next, execution paths are clustered together to create abstract code summary of the target subject systems. Feng \textit{et al.}~\cite{feng2018hierarchicalExecutionComprehension} also named the clusters by extracting keywords from the function names present in execution paths. In their study, only the TFIDF technique is applied to extract and name intermediate clusters.

For this study, our motivation is to take forward this approach and enrich it with existing techniques from the literature. Two limitations of the study from Gharib \textit{et al.}  are using only TFIDF method for information retrieval and no presence of user study to validate how developers prefer the output abstractions. We adopt two more topic modeling techniques for information retrieval, which show promising results for naming source code artifacts in the literature \cite{de2012IRMethodsArtifacts}. Andrea \textit{et al.}  \cite{de2012IRMethodsArtifacts} tried to apply IR techniques like VSM, LDA, and LSI on  source code artifacts. To evaluate IR techniques' effectiveness, they also produced suggestions from 17 users on the same classes. Then, they assessed the performance of automatic naming by comparing overlap
with manual naming of users. In their study, authors also find that heuristic based approaches focusing on function signatures perform well for code artifacts summarization. Inspired from their study, we use LDA and LSI on function signatures to extract concepts in code in this study. Another improvement from Gharib \textit{et al.} is to adopt a user study for validating automatic abstraction. Sonia \textit{et al.}  \cite{haiduc2010supporting} used Pyramid score to evaluate the output of automatic code summary with developers' summary. We also adopt this Pyramid score, which is widely used for the evaluation of natural language summaries.








% \chapter{Related Work}
% In this chapter, we discuss relevant literature in regard to program comprehension, reverse engineering and feature location. In Section \ref{related:program_comprehension}, we discuss scope of program comprehension from literature. We then present how different IR techniques used for naming source code artifacts in Section \ref{related:IR}. Last, in Section \ref{related:reverse_engineering} and \ref{related:feature_locate}, we explained different reverse engineering techniques and how developers locate features.


\section{Program Comprehension in general}
\label{related:program_comprehension}
Program comprehension is a cognitive way of understanding software systems to perform different software maintenance tasks \cite{wei2002surveyCategorizationComprehension, siegmund2016programPastFuture}. Three different type of cognitive models \cite{tilley1998reverseEngineeringFramework, von1993programToolRequirements, siegmund2016programPastFuture} can be found in the literature which is followed consciously or unconsciously by developers. The comprehension models are Top-down, Bottom-up, and Integrated. When developers have prior domain knowledge about a software system, the top-down model is preferred as they can map domain knowledge to low-level source code hierarchically \cite{brooks1983theoryComprehensionPrograms}. On the other hand, when developers lack domain knowledge, they start with low-level source code and then group the functionality together to have a hierarchical abstraction of the system features \cite{shneiderman1979syntacticInteractionsModel, pennington1987stimulusMentalRepresentations}. Integrated model \cite{shaft1995relevanceDomainKnowledge, von1993programToolRequirements} is a mix of top-down and bottom-up approach. The problem in hand and the target system have different properties in the real world, which demand switching between top-down and bottom-up models. Generally, a developer can have prior domain knowledge of a few portion and point-blank for the rest of the system. This situation deserves the adapted use of both top-down and bottom-up approaches.   


\section{IR techniques to Name source code artifacts}
\label{related:IR}
% \cite{mcburney2014improvingTopicSummarize}
% \cite{de2012IRMethodsArtifacts} \cite{panichella2013topicModelsTasks}
% \cite{chen2016topicMiningRepositories}
% Very very important \cite{sun2016surveyTopicSE}
As software repositories contain unstructured data, topic model techniques are widely applied for different software engineering tasks to retrieve information \cite{chen2016topicMiningRepositories, panichella2013topicModelsTasks, sun2016surveyTopicSE}. Most common tasks where topic models showed promising results are source code comprehension, feature location, refactoring, bug localization, and others \cite{sun2016surveyTopicSE}. Lucia et al. \cite{de2012IRMethodsArtifacts} conducted a study to see how information retrieval techniques perform compared to manual naming Java class files. Developers were asked to pick ten keywords for each class file, and top-10 words are picked using different topic model technique and custom heuristics. Their experiment shows that in 40\%-80\% cases, automatic and human label overlaps. 

\section{Reverse Engineering}
\label{related:reverse_engineering}
\subsection{Subsystem Identification}
Muller et al. \cite{muller1990composingSubsystemStructures} proposed subsystem detection algorithm using different clustering components like variable, procedure, and modules. 
According to Bass et al. \cite{bass2003softwareArchitecturePractice}, two types of software architecture are useful for understanding a complex software system. They are Conceptual and Concrete architecture. A conceptual architecture provides high-level abstraction skipping the code level details. On the other hand, concrete architecture shows the implementation level information. Roy et al. \cite{roy2008softwareArchitectureRecovery} propose and evaluate a framework for the incremental and iterative application of automated architecture recovery (using SWAG Kit) and architecture analysis (using SAAM.). They showed that the reverse engineering tool cannot recover a deeply understood conceptual architecture without SAAM's application but can create a reasonable basis towards that direction. Murphy et al.\cite{MurphyNotkin2001} show that by generating reflexion models from high-level model and source model (i.e., static call graphs), it is possible to facilitate program understanding to the novice developers. 

In this study, we try to automatically recover conceptual architecture from concrete architecture, reducing manual effort.

\subsection{Call graphs to abstract a software system behaviors}

Static and dynamic call graphs are used in literature to help developers comprehend a software system to aid different maintenance tasks \cite{feng2018hierarchicalExecutionComprehension, gharibi2018automaticStaticCluster, xin2019identifyingFeaturesExecution}. Feng et al. \cite{feng2018hierarchicalExecutionComprehension} proposed an approach to use dynamic call graphs for understanding a system's behavior. They instrumented the subject systems to generate execution traces of method entry and exit events. Later, they followed the duplication removal process and constructed a call graph from the execution traces. Execution phases from this dynamic call graph are clustered to get system behaviors. Similarly, Gharib et al. \cite{gharibi2018automaticStaticCluster}, and Vijay et al. \cite{walunj2019graphevoEvolutionCall} also adopted clustering of execution paths from call graphs of the static variant. Using a static call graph brings the benefit of capturing all possible scenarios and less redundant data to handle than dynamic call graph \cite{gharibi2018automaticStaticCluster}. 

\subsection{IR techniques on the hierarchical abstraction of software system}
Feng et al. \cite{feng2018hierarchicalExecutionComprehension} proposed an approach to identify behaviors of a system by hierarchically abstracting dynamic call graph from execution traces. Sequential pattern mining is applied to mine significant portions from the execution phases. Hierarchical clustering is performed to group execution phases. Later, the clusters are labeled using the TFIDF score, where method signatures serve as terms and the phases as document. 
Paul et al. \cite{mcburney2014improvingTopicSummarize} used static call graph to hierarchically abstract a system. In their hierarchical view, each node represents a method. To mine the topics, keywords from methods are considered. Hierarchical Document Topic Model (HDTM) by \cite{weninger2012documentTopicHierarchies} Weninger et al. is adopted, which works on graph documents to mine topic. Gharib et al. \cite{gharibi2018automaticStaticCluster} took a different approach. They went further with the static call graph by extracting execution paths and then clustering the execution paths. Each cluster in the cluster tree is labeled using top-5 method names from Tfidf. Levy et al. \cite{levy2019understandingLargeHierarchical} found interviewing developers that two kinds of comprehension go for large scale hierarchical view. They are system comprehension and code comprehension. In this paper, we tried to adopt static call graph analysis from Gharib et al. and then improve their labeling technique. Nodes of the cluster tree is considered as a behavioral abstraction unit of a system. Method comments are used to generate a description of the unit and sequential pattern mining to create sample examples. 

\section{How developers locate feature on source code}
\label{related:feature_locate}
Kruger et al.~\cite{kruger2019features} studied on two data sets (67 developers IDE activity, 600 developers IR-based tool usage). They suggested that there is room for improvement in the existing code navigation, code search tools. The manual process followed by developers to locate feature are of mostly three types ~\cite{damevski2016field, wang2011exploratory, revelle2005understanding}. First, developers use information retrieval based tools to query for feature related keywords. In this thesis, we have used IR based techniques to label nodes. Developers can use our tool to find keywords of their interest. Second, execution-based process where developer try to find execution scenarios where the feature is active. After finding relevant execution scenarios, developers debug the execution scenarios by setting breakpoints. In our second study, we have attached execution patterns to nodes which can be utilized by developers to know where to set the breakpoints for understanding a feature. Third, exploration based process where developer explore source code to understand method calls to find a feature. In HCPC tool, we showed method execution patterns for each node. Our tool can also help developers for browsing code using exploration based process.
% Include this paper and its related works. \cite{damevski2016field} \cite{kruger2019features} see 5.5.4 process section 

\chapter{ Labeling abstraction nodes and human evaluation}
\label{chapter:hla1}

In this chapter, we discuss our approach compared to existing studies for labeling abstraction nodes. In Section \ref{hla1:intro} and \ref{hla1:motivation}, we introduce important concepts, related works and what we did to advance them. Section \ref{approach} presents our approaches for cluster naming, Section \ref{Experimental} describes our experimental design, 
Section \ref{results} presents the technique evaluations,
and finally, Section \ref{conclusion} summarizes the chapter by mentioning our future direction.  


\section{Introduction}
\label{hla1:intro}
Understanding the source code of a software system is a prevalent and vital task for the developers because many software engineering tasks depend on program comprehension \cite{cornelissen2009systematic, gilmore1991models, xie2016revisit, feng2018hierarchicalExecutionComprehension}. It is difficult for an individual developer to develop an enterprise software system on their own. Therefore, when someone is assigned to a task or join a development team, they need to understand the existing system to get used to the system. This program comprehension involves a lot of browsing back-and-forth between different granularity levels of the codebase. To reduce developers' effort to comprehend program artifacts, a lot of research is going on in the field of program comprehension \cite{feng2018hierarchicalExecutionComprehension, gharibi2018automaticStaticCluster, kulkarni2014supporting, izu2019program}. An abstract representation of the target software system can easily guide the exploration of low-level source code depending on developers' maintenance tasks. One of the approaches is to generate dynamic logs of function executions while running an existing system on different test cases. The logs can then be used with other methods to produce a suitable output for developers to comprehend the software system \cite{feng2018hierarchicalExecutionComprehension}.
Moreover, most of the dynamic approaches generate dynamic call graphs from the generated dynamic logs of various system scenarios. However, the problem with dynamic call logs is that they only consider the function executions invoked during the dynamic log generation of a target system based on the test cases. As a result, not all the functionalities of the target system are considered during the codebase investigation. Another problem with dynamic logs is that they generate billions of data points, which are mostly redundant. If someone wants to abstract the whole system for comprehension purposes, then using dynamic logs does not help much to cover the entire system.
On the other hand, a static call-graph can be generated by extracting caller-callee relationships from source files. The benefit of a static call graph is that it is possible to have a target system's overall functionalities. The static call graph also resolves the problem of redundant data of the log generations.



A large portion of a developers' development time is devoted to understanding existing source codes \cite{corbi1989program, minelli2015know, ko2006exploratory}. Because without knowing the cognitive relation between source code with higher-level system functionalities, it is difficult to perform different software maintenance tasks (e.g., debugging, feature addition, refactoring, and testing). So, browsing back-and-forth between different source files of a system is widespread among developers to comprehend an existing system. What developers usually do is that they first look for the name of a source file's functions to understand the intention of the functions \cite{de2012IRMethodsArtifacts, starke2009searching}. Therefore, the function names can be utilized for abstracting a system's higher-level functionalities. Moreover, existing studies suggested that \cite{salah2005scenariographerReverseEngineering,pradel2009automaticUseageSpecification} sequence of function invocations can help extract usage scenarios or higher-level functionalities of a target system. Hence, having a tool that visualizes the cognitive mapping between source code and high-level functionalities and allows browsing through source code in a more informed way would help developers.

% \subsection{Why is it hard? (E.g., why do naive approaches fail?)}
% complex source code base, difference between natural language and source code, different characteristics of different language and frameworks, huge amount of data to summarize, creating priority
Manually browsing source code for locating concepts is a laborious task. As a consequence, a lot of existing studies have been done to map concepts with source code using dynamic execution logs. However, very few studies considered static call graphs and emphasized function names. Gharib \textit{et al.} \cite{gharibi2018automaticStaticCluster} proposed a technique based on static call graphs where concepts are mapped with source codes. The authors have presented a whole subject system as a tree where nodes represent concepts of the system. However, they have only applied the TFIDF technique to extract the concepts of a particular codebase. During concept location, they have just considered the name of the function as term. Another drawback of their study is that they have not conducted any use case analysis from users' perspectives. So how developers will be comprehending the source code of a software system is absent in the study for real-world cases.

These limitations of the existing work motivated us to investigate more details on the potential of this approach. We have applied one information retrieval technique, TFIDF, and two topic modeling techniques (LDA and LSI). In the previous study \cite{gharibi2018automaticStaticCluster}, the full function name is treated as a term for the TFIDF technique. Here, we introduce words in function names as another variation. In total, we have six techniques to evaluate, as each technique mentioned above has two variations (function name and words in function name) result. We have also performed a small scale user-study with five developers. We have used 12 clusters from three subject systems as use cases to evaluate our approaches. Developers have rated the summaries generated by each technique and provided their summary of each use case, which we used to assess our automatic techniques using the Pyramid metric. From our investigation, we have found that automatic labeling using TFIDF for words in method names as term variation has an average of 64\% overlap with manual labeling of participants. LDA and LSI received 37\%, and 23\% overlap accordingly. We have also found that words in function name variants got a minimum of 5\% more preference rating compared to function name variants from developers. 

In summary, our contributions are:
\begin{itemize}
  \item We adopted two topic modeling techniques to name nodes of the abstract code summary tree. 
  \item We introduce using words in the function name as a term for information retrieval techniques.
  \item We have conducted small scale user study to evaluate the proposed techniques.
\end{itemize}



\section{Motivational Example}
\label{hla1:motivation}

This section is presented with a motivational example of real-world scenarios. Suppose Bob joined a new company X as a Junior Software Developer. He needs to work on a software project which is being developed for more than six years. He must have a cognitive mapping between source code artifacts and high-level concepts of the software project, which will boost his integration to the project. To get an understanding of the project, he can use the Call graph of the project, which visualizes functional dependency.

\begin{figure*}[h]
  \centering
  \includegraphics[width=\columnwidth]{figures/hla1/realTime.png}
  \caption{A portion of the Call graph of Real-Time-Voice-Cloning project by Pyan}~\label{fig:realTime}
\end{figure*}

However, in Figure \ref{fig:realTime} we can see a portion of the large call graph generated using Pyan~\cite{pyan} for Real-Time-Voice-Cloning~\cite{realTime} project. This presentation is very complex and hard to comprehend. Furthermore, if Bob has any particular Software Engineering task to do, first, he needs to locate the concept in source code. Locating source code artifacts relevant to the specific task will help Bob do his task faster. Therefore, our approach starts from this complex call graph and extracts concepts from execution paths in various hierarchical levels. Using the proposed approach, Bob can explore concepts from top-to-bottom, which at the end map to execution paths and the name of functions for smooth inquiries. 


 
 
% \vspace{6mm}
\section{Approach}

\begin{figure*}[tb]
  \centering
  \includegraphics[width=\columnwidth]{figures/hla1/visual_tool_static_call_graph-2.png}
  \caption{Structure of an abstract code summary tree}~\label{fig:hla1_motivation}
\end{figure*}
In this section, we discuss two significant steps in our approach with a brief discussion. First, in Section \ref{hla1:approach_acs}., we described six steps to get the cluster tree of a subject system. Second, in Section \ref{hla1:node_title}, we explain how we used different information retrieval techniques to label nodes of the abstract code summary tree. Data collection for evaluating the approach is depicted in algorithm \ref{hla1:alg:overall}.

\label{approach}

\begin{algorithm}
    \SetKwInOut{Input}{Input}
    \SetKwInOut{Output}{Output}
    
    \underline{Call Graph to abstract code summary tree} $(call graph)$\;
    
    \Input{Call graph}
    \Output{Abstract code summary tree}
    \For{Iterate each node in the call graph}
    {
        \If{ $Number\_of\_Incoming\_Degree(node) == 0$}
        {
            entryNodes.append(node);
        }
        \If{$Number\_of\_Outgoing\_Degree(node) == 0$}{
            exitNodes.append(node);
        }
    } 
    \For{$i\gets1$ \KwTo $entryNodes.length$ \KwBy $1$}
    {
        \For{$j\gets1$ \KwTo $exitNodes.length$ \KwBy $1$}
        {
            execution\_paths.append($simple\_DFS\_path(i, j)$);
        }
    }
    \For{$i\gets1$ \KwTo $execution\_paths.length$ \KwBy $1$}
    {
        \For{$j\gets1$ \KwTo $execution\_paths.length$ \KwBy $1$}
        {
            $distance\_matrix[i][j]$ = $consine\_similarity(i,j)$;
        }
    }
    $cluster\_tree$ = $create\_cluster\_tree(distance\_matrix)$;
    
    $abstract\_code\_summary\_tree$ = $generate\_label\_for\_each\_node(cluster\_tree)$;
    
    return $abstract\_code\_summary\_tree$;
    \caption{Constructing Python source code to an abstract code summary tree}
    \label{hla1:alg:overall}
\end{algorithm}

% \vspace{4mm}
\subsection{Abstract Code Summary (ACS) Tree}
\label{hla1:approach_acs}
The call graph is a visual representation of the relationships between the functions of a project. We adopt static call graphs, which are generated by analyzing source code. As the static call
graphs capture all function calls of a target system, we
choose to abstract the target system. Previous studies suggested that function names contain significant abstraction of source code. Thus, we emphasize mining concepts by analyzing function names in the static call graph.
As we want to capture and abstract the overall system's high-level concepts, therefore, the decision for adopting a static call graph as a building-block of our approach and using function names for concept location is well-justified.  

In Figure \ref{fig:hla1_motivation}, we present the structure of our proposed abstract code summary tree. The leaf nodes of this tree are directly mapped to the execution paths. The execution paths are a list of function names executed sequentially during the execution of a software system. For instance, node 5 is mapped to the execution path where \texttt{ F11, F6, F7, and F9} are called sequentially. Similarly,  in this scenario, all the four-leaf nodes 4, 5, 6, and 7 are mapped to four execution paths or function call sequences. Node 1, 2, and 3 are intermediate nodes of the tree. Naming these intermediate nodes analyzing the execution paths that resides under them might reduce the need to go through in detail about their functionalities. In the figure, node 2 has been named \texttt{F11\_F6}, and node 3 has been named as \texttt{F3\_F1}  by analyzing the function names in the execution paths under those nodes. If we find a proper naming technique that can map concepts in source code with different granularity levels, this approach can make developers program comprehension tasks more flexible. In Figure \ref{hla1:fig:overall}, all the steps are visualized to generate ACS tree from source code.
  

\begin{figure*}[tb]
  \centering
  \includegraphics[width=\columnwidth]{figures/hla1/visual_tool_static_call_graph.png}
  \caption{Overview of the overall approach}~\label{hla1:fig:overall}
\end{figure*}

\subsubsection{Analyzing source code using modified Pyan module}

For extracting function relationships from a python system, we used a modified version of Python module Pyan \cite{pyan}. Pyan works only for a single directory. We adapted Pyan so that it can consider multiple directories while extracting the relationships. Pyan uses the abstract syntax tree (AST) for extracting function relationships. After analyzing the source code, we generated a graph in TGF (Trivial Graph Format). In TGF, all modules and functions' physical addresses in the source code are printed first. Then, relationships between all functions are presented as the caller and callee pair.

\subsubsection{Extracting function relationships from TGF}

Function relationships from the TGF file are used as inputs in our technique. Encoded unique identifiers are used to replace function names for ease of processing during the hierarchical clustering step.

\subsubsection{Static call graph creation based on the extracted relationships}
To perform different graph operations, we have created graph objects of the NetworkX \cite{networkx} module using the extracted function relationships. 
\subsubsection{Extracting execution paths}

The execution path is a simple path between an entry node and an exit node. An entry node is a node in the call graph which incoming edge degree is zero. Hence, no function is dependant on an entry node. An exit node is a node that has a zero degree of outgoing function calls. We have generated a list of entry and exit nodes to generate execution paths from a call graph. A simple path means no repeated node visit while visiting from the source node to the destination node. We have collected all possible simple paths for all possible combinations of entry node and exit node pairs. We have implemented a simple path finding algorithm from the NetworkX library, which uses a modified DFS algorithm for finding simple paths between a pair of nodes \cite{networkx}. For our task, a source node is an entry node, and a destination node is an exit node.    

\subsubsection{Distance matrix for execution paths}

For clustering execution paths (sequence of function names), we need to measure the similarity between all pairs of execution paths. For this purpose, we implemented the Jaccard similarity measure \cite{niwattanakul2013jaccardKeywordsSimilarity}. The linkage algorithm uses this similarity in the next step. If we have two sets $ A $ and $ B $, then their Jaccard similarity will be the ratio of their intersection's cardinality by the union. The clustering algorithms work on the distance, which is, in our case, the dissimilarity between two execution paths/clusters. We have subtracted the similarity score with one to get the dissimilarity value according to equation \ref{eq:jaccard}. After calculating dissimilarity between all pairs of execution paths, we converted the 2d matrix to 1d condensed matrix to make our program memory efficient.

\begin{equation}
Dis(A, B) = 1 - \frac{A\cap B}{A\cup B}
\label{eq:jaccard}
\end{equation}

\subsubsection{Clustering execution paths using linkage algorithms}

To group similar execution paths as clusters, we have implemented a linkage algorithm using popular python package Scipy \cite{scipy}. Scipy has different types of linkage algorithms already implemented in its core. To update the distance between two clusters, we have picked Ward the minimum variance method \cite{ward}. Equation \ref{eq:ward} shows how distance using the Ward method is updated when two clusters from cluster forest are merged into a new one \cite{scipy}.

\begin{equation}
     d(u, z) =   \sqrt{\frac{(n_x+n_z)d(x,z)^2+ (n_y+n_z)d(y,z)^2 - n_z d(x,y)^2 }{n_x+n_y+n_z}}
    \label{eq:ward}
\end{equation}

 
In equation \ref{eq:ward}, $u$ is a newly formed cluster, and $z$ is an unused cluster which will be used as reference to calculate distance. $n_x$, $n_y$ and $n_z$ are respectively the number of execution paths (as we are clustering the execution paths) in cluster $x$, $y$ and $z$.
When a new cluster $u$ is created, the distance between $u$ and all the other clusters are updated in the distance matrix. Additionally, cluster $x$ and $y$ are removed from the distance matrix as they have been merged as a new cluster $u$. This step is followed iteratively until only a single cluster remains in the cluster forest. 

For example, in Figure \ref{fig:motivation}, initially, at the start of the clustering process, there are four clusters 4, 5, 6, 7. Next, the hierarchical clustering algorithm selects the two most similar clusters (4, 5 ) to merge them as a new cluster 2. Now, in the clustering process, we have three clusters 2, 6, 7. Similar to the previous step, the most two similar clusters are merged into one. This process continues until there is only one cluster left. Ward method is used to calculate distance between the newly merged cluster with others.

% \vspace{4mm}

\subsection{Naming Nodes in an Abstract Code Summary Tree}
\label{hla1:node_title}
After getting a cluster tree from the previous step, our next step is to name the clusters to represent the high-level functionality of source code in a readable way. In this step, we will be able to locate high-level concepts in the ACS tree. However, each cluster has a list of function call sequences, and the function call sequences are called execution paths. Our challenge is to extract essential keywords from this collection so that developers can get an overview of the underlying high-level functionalities under the cluster. Naming the source artifacts correctly, in our case, which is nodes in the abstract code summary tree, is the fundamental contribution of this work. Proper naming can help developers to comprehend a program promptly. Toward the naming, we have applied three popular techniques used widely in natural language summarization tasks. These methods try to find meaningful and significant topics from a set of documents. In our approach, a document is an execution path that contains a list of function names. All the execution paths under a cluster are considered as documents.
A previous study used function names as terms in a document \cite{gharibi2018automaticStaticCluster}. However, we want to see what happens if we parse the function names and use the words in function names and use them as a term in documents. We used both words in a function name, and method names approach for the three techniques. In Section \ref{background:techniques}, we have discussed TFIDF, LDA and LSI techniques to generate node label.
% Below we briefly described how these three techniques work. 

% \subsubsection{TFIDF}
% TFIDF \cite{ramos2003usingTfidfRelevance} is a very popular information retrieval technique widely used in text-based search engines. The full form of the TFIDF is term frequency-inverse document frequency. Term frequency means how frequent a term in a document is. Term frequency is calculated according to equation \ref{eq:tf}.

% \begin{equation}
%     tf(t,d) = f_{t,d}
%     \label{eq:tf}
% \end{equation}
% \begin{equation}
%     idf(t) = \log(\frac{n}{df(t)})+1
%     \label{eq:idf}
% \end{equation}
% \begin{equation}
%     tf-idf(t,d) = tf(t,d) * idf(t)
%     \label{eq:TFIDF}
% \end{equation}
% The function $f_{t,d}$ counts frequency of term $t$ in the document $d$. Inverse document frequency is calculated according to equation \ref{eq:idf}. Function $df(t)$ in equation \ref{eq:idf} is the count of documents term $t$ is present. The main purpose of $idf$ is to penalize common keywords in the corpus. Term frequency (tf) and Inverse document frequencies (idf) are multiplied to get score for terms. We have adopted \texttt{TFIDFVectorizer} class of scikit-learn \cite{scikit-learn} library for implementing $TFIDF$ technique.
% \subsubsection{LDA}
% Latent Dirichlet Allocation (LDA) \cite{blei2003latentLDA} is a statistical model that tries to describe a set of documents by assuming they are created from some topics. LDA is a very popular topic modeling technique. LDA assumes every term in a document belongs to some topic. So, it assumes each term belongs to some topic and then performs analysis to find which assumptions are supported by statistics of the corpus. We have used Gensim \cite{gensim} library for implementing LDA for our approach.
% \subsubsection{LSI}
% Latent Semantic Indexing (LSI) \cite{deerwester1990indexingLSI} is a technique used in natural language processing. LSI assumes semantically similar words occur together. First, the term-document frequency matrix is calculated from the corpus. Then, this term-document frequency matrix is decomposed into three matrices using the Single Value Decomposition (SVD) technique. Terms are first assigned to topics using the term-document frequency matrix. Then, using all the topics, a topic importance matrix is derived, which leads to topics for the documents. Similar to LDA, we used Gensim \cite{gensim} library for implementing LSI. 

\begin{figure}
    \begin{center}
    \subcaptionbox{Full abstract code summary tree with local view\label{fig:tool_ui}}
    {\includegraphics[scale=0.25]{figures/hla1/ToolUI.png}}
    \subcaptionbox{Form presented to the participants for answering\label{fig:userstudy}}
    {\includegraphics[scale=0.23]{figures/hla1/userstudy.png}}
    \caption{Tool UI presented to the study participants}
    \label{fig:tool}
    \end{center}
\end{figure}

\section{Experimental Design}
\label{Experimental}
This section will discuss the research questions that need to be answered regarding the abstract code summary tree, how we collected our subject systems for the experiment, and details about users who participated in this study.


\subsection{Research questions}
We want to explore how manual naming supports automatic naming techniques. To investigate this, we set \texttt{RQ1} described below.
Besides, we have compared developer preferences for three different techniques using function names as terms by \texttt{RQ2}. Similarly, for \texttt{RQ3}, we changed the input for information retrieval techniques by words in function names instead of function names and compared developers' ratings among the three approaches. Finally, we want to see the performances of our two variations of choosing terms by a systematic comparison by \texttt{RQ4}. The four research questions correspond to the overarching research questions 1, 2 described in Section \ref{intro:research_questions}.  



\begin{itemize}
    \item \textbf{\texttt{RQ1}} How well does the automatic labeling perform using the candidate approaches compared to manual labeling? 
    \item \textbf{\texttt{RQ2}} How do developers evaluate different labeling approaches based on function names?
    \item \textbf{\texttt{RQ3}} How do developers evaluate different labeling approaches based on words in method names? 
    \item \textbf{\texttt{RQ4}} How can we compare the preferences of developers between the two approaches addressed in \texttt{RQ2} and \texttt{RQ3}? 
\end{itemize}



\subsection{Dataset Collection}
In order to conduct the user-study, we have collected source code of three popular Python projects \emph{Detectron}~\cite{Detectron2018}, \emph{Real-Time-Voice-Cloning}~\cite{realTime} and \emph{requests}~\cite{requests}. The reason behind choosing these subject systems for our study is that they are popular among Python developer communities. These projects follow the standard conventions of software developments so participants will be able to relate keywords from their day-to-day knowledge. Additionally, open-source projects tend to follow proper function naming conventions, which is important for our approach as it completely depends on function names. We extracted source code and applied the steps described in Section \ref{alg:overall}. We have printed clusters with their corresponding execution paths and names suggested by the candidate techniques in a file for doing the user-study. We have chosen 12 clusters semi-randomly, i.e.,  four from each of the subject systems, which ensures the coverage of different levels' clusters. 

\begin{table}[h]
\small
\caption{Pyramid score computation }
\label{table-pyramid1}
\centering
\begin{tabular}{|l|l|l|l|l|l|l|l|l|l|l|l|l|}
\hline
  & response & request &	dict &	send &	from &	build &	cookiejar &	create &	get &	cookie &	prepare &	merge  \\ \hline
D1 & x      & x     & x           & x   & x    &        &         &     &      &    &  &    \\ \hline
D2 &     x   &      &  x       &     &    & x      & x       &   &      & x& &       \\ \hline
D3 & x      &       &           &    &     &    x    &         & x   &  x    & x & &     \\ \hline
D4 &        &       &             &   x  &     &        &        &     & x    & x    & x &    \\ \hline
D5 &  x     &      &    x         &     &     &        &        & x    &     & x    &  &   \\ \hline
\texttt{TFIDF\_word}  &   x(4)     &       &             &  x(2)   &  &        &     &     &  &  & & x(1)    \\ \hline
\texttt{LDA\_word}  &        &   x(1)    &             &     &  &        &     &     & x(2) &   & x(1) &  \\ \hline
\texttt{LSI\_word}  &        &   x(1)    &             &  x(2)   &   &        &  x(1)     &     & x(2) &  & & \\ \hline
\end{tabular}
\end{table}

\subsection{User-study}
For the 12 clusters, we manually analyzed each cluster's execution paths and come up with a 2-3 line description of what happens inside the clusters. Before the study, we told users to rate the automatic summaries of our three techniques, each with two variations. Additionally, we also provided a text box for the participants to select five keywords from the description produced by manual analysis of execution paths. We have used this summary to compute the Pyramid score for the three techniques of the words in function name variation. A total of five persons participated in the study with a software development background. Among them, three are female, and two are male. Each of the participants has at least a Bachelor's degree in Computer Science. Two of them are graduate students, and the other three are working as developers in three different software firms. All of them have at least three years of experience in programming experience with an average of 3.8 year. 


In Figure \ref{fig:tool_ui}, the abstract code summary tree from our tool is presented. The upper box contains the full abstract code summary tree. Below the concept tree, a local view can be used to look closer to the concept cluster. Developers can click on any node in the concept diagram to get a zoomed view of its child and parents. 

In Figure \ref{fig:userstudy}, a screenshot of the form provided to the participants is presented. When participants right-click on the target clusters, a form with cluster id and a brief description of the execution paths' manual analysis is popped up. In the form, we asked the participants about their preference (1 means least preferred, 5 means most preferred) for names suggested by the six techniques and selected five keywords from the descriptions to complete the study.




\section{Results and Discussion}
\label{results}
% \vspace{3mm}

\begin{figure*}[h]
 \pgfplotstableread{
 1 0.7857 0.71 0.29
2 0.71 0.29 0.14
3 0.53 0.06 0.06
4 0.56 0.125 0.125
5 0.61 0.46 0.15
6 0.57 0.05 0.315
7 0.9 0.72 0.09
8 0.53 0.53 0.46
9 0.43 0.31 0
10 0.46 0.26 0.4
11 0.81 0.5 0.43
12 0.73 0.4 0.33
}\dataset
% \resizebox{6.0in}{!}{
\begin{tikzpicture} \begin{axis}[xbar,
ylabel={\#Cluster no.},
xlabel={Pyramid score},
 symbolic y coords={1, 2, 3, 4, 5, 6, 7, 8, 9, 10, 11, 12},
        ytick=data,
        nodes near coords, 
        nodes near coords align={horizontal},
        bar width=0.18cm,
width=\textwidth,
% height=7.5cm,
xmin=0,
xmax=1.0, 
xticklabel style={font=\footnotesize},
yticklabel style={font=\footnotesize, /pgf/number format/fixed},
y tick label style={ rotate=0},
major y tick style = {opacity=0},
minor y tick num = 1,
minor tick length=1ex,
xmajorgrids = true,
%grid=both,
%grid style={line width=.1pt, draw=black!10},	
legend entries={TFIDF\_word,LDA\_word, LSI\_word},	
legend style={at={(0.5,-0.1)},
anchor=north,
legend columns=6,
font=\scriptsize,
% text width=2.75in,
% minimum height=0.20in,
nodes near coords
}
        ]
    \addplot [draw=black!100, fill=black!10] table[x index=1,y index=0] \dataset;
    \addplot [draw=black!100, fill=black!30] table[x index=2,y index=0] \dataset;
    \addplot [draw=black!100, fill=black!50] table[x index=3,y index=0] \dataset;
    
\end{axis}
\end{tikzpicture}
% }
  \caption{Pyramid score of the 12 clusters}~\label{fig:pyramid12}
\end{figure*}


\subsection{User Naming vs. Automatic Naming}

To investigate how automatic naming accords with manual naming, we have used Pyramid score \cite{nenkova2004evaluating}. Pyramid score is used in natural text summarization tasks to compare an automatic summary with a manual summary. Haiduc et al. \cite{haiduc2010supporting}  used Pyramid score to support source code summary with developers' summary, which motivated us to adopt Pyramid score to find out how our automatic approaches of abstraction harmonize with developers' selections. In Table \ref{table-pyramid1}, we have shown the Pyramid score calculation process for a cluster (i.e., cluster number 10 of the 12 clusters). The preferences of five developers who participated in this study are represented by D1,\ldots, D5. X\_word represents the corresponding outputs of $X \in {TFIDF, LDA, LSI }$ by considering words in function names. Each column presents unique keywords from the selections of five participants. Furthermore, we have marked which words are matched with the automatic summary from a developers' summary in the corresponding cells. In each row for the automatic techniques, we have put the support from five developers for keywords being present in the summary.
\begin{figure}[h]
\pgfplotstableread{
 1	2.6	3	3
 2	3.4	2.8	4
 3	2.8	2.8	3.4
 4	3	3.2	3.2
 5	3.2	3	2.6
 6	3.2	2.2	2.8
 7	3.2	2.8	3.2
 8	3.4	2.8	3.2
 9	3.2	3.4	3.4
 10	3	3.4	3.6
 11	3	3.8	3.2
 12	3.6	3.4	3
}\dataset
\resizebox{6.0in}{!}{
\begin{tikzpicture}
\begin{axis}
[ybar,
% enlargelimits=0.05,
bar width=0.10cm,
width=0.55\textwidth,
height=4.5cm,
ymin=0,
ymax=5.0, 
ylabel={Average ranking of
five participants},
xlabel = {\#Cluster no.},
yticklabel style={font=\footnotesize},
xticklabel style={font=\footnotesize, /pgf/number format/fixed},
xtick=data,
xticklabels = {1, 2, 3, 4, 5, 6, 7, 8, 9, 10, 11, 12},
x tick label style={ rotate=45},
major x tick style = {opacity=0},
minor x tick num = 1,
minor tick length=1ex,
ymajorgrids = true,
label style={font=\footnotesize},
%grid=both,
%grid style={line width=.1pt, draw=black!10},	
legend entries={TFIDF(function names),LDA(function names), LSI(function names)
},	
legend style={
at={(0.5,-0.35)},
anchor=north,
legend columns=-1,
font=\scriptsize,
%text width=2.75in, 
minimum height=0.20in,
nodes near coords style={rotate=90,  anchor=west, font=\tiny},
nodes near coords
},
]
% \addplot[draw=black!100, fill=green!20,pattern= grid] table[x index=0,y index=1] \dataset;
\addplot[draw=black!100, fill=black!10] table[x index=0,y index=1] \dataset;
\addplot[draw=black!100, fill=black!30] table[x index=0,y index=2] \dataset;
\addplot[draw=black!100, fill=black!50] table[x index=0,y index=3] \dataset;
\end{axis}
\end{tikzpicture}
}
\caption{User preference among three implemented naming techniques (considering methods as terms)}
\label{fig:method}

\end{figure}


For example, we can see that keyword $response$ is present in TFIDF with words in the function name variation, and four of the developers picked $response$ in their summary. So, support for keyword $response$ is given 4. To get the Pyramid score for cluster number 10, we have summed each keyword's support in automatic naming by developers. In this case, values are $4(response), 2(send), 1(merge)$. We divide the sum of these support values by the top five most frequent keywords of five developers' summary. So, the score is now $(4+2+1)/(4+4+3+2+2) = 0.466 $ for cluster 10. Greater Pyramid score means that the automatic naming is becoming more human in our case. In Figure \ref{fig:pyramid12}, we have plotted Pyramid score for 12 clusters with the three techniques of word variant and support of the five participants for them. In the figure, we can see that for most of the clusters, the \texttt{TFIDF\_word} based automatic naming technique's summary agrees more compared to other techniques with  the developers' provided summaries.



% \vspace{3mm}
\subsection{User Rating on Function Name Variant}
To answer the RQ2, we use the techniques with function names as unit variation. We asked our participants to rank each technique's summary with a score ranging from 1 to 5 to reflect how well they support the manual description. In Figure \ref{fig:method}, we have plotted the average ranking of the participants for 12 clusters with the techniques. In the figure, we can see the users preferred LSI naming technique over the LDA. LSI is preferred in almost 50\% of the clusters. For clusters 1, 4, 9, participants' preference for LDA and LSI are the same. The reason is that both techniques provided a similar kind of summary for the cluster in the automatic naming process.



\begin{figure*}[h!]
\pgfplotstableread{
1	3.4	3.4	3
 2	3.4	3.2	3
 3	3	3.2	2.8
 4	3.2	3	3.2
 5	3.4	3.8	3
 6	3.6	3.4	3.8
 7	3.6	3.6	3.6
 8	3.2	3	3.6
 9	4	3.4	4.2
 10	3.6	3.6	4
 11	3.8	3.6	3.8
 12	3.8	3.8	3.4
}\dataset
\resizebox{6.0in}{!}{
\begin{tikzpicture}
\begin{axis}
[ybar,
% enlargelimits=0.05,
bar width=0.10cm,
width=0.55\textwidth,
height=4.5cm,
ymin=0,
ymax=5.0, 
ylabel={Average ranking of
five participants},
xlabel = {\#Cluster no.},
yticklabel style={font=\footnotesize},
xticklabel style={font=\footnotesize, /pgf/number format/fixed},
xtick=data,
xticklabels = {1, 2, 3, 4, 5, 6, 7, 8, 9, 10, 11, 12},
x tick label style={ rotate=45},
major x tick style = {opacity=0},
minor x tick num = 1,
minor tick length=1ex,
ymajorgrids = true,
label style={font=\footnotesize},
%grid=both,
%grid style={line width=.1pt, draw=black!10},	
legend entries={TFIDF(words in function names),LDA(words in function names), LSI(words in function names)
},	
legend style={
at={(0.5,-0.35)},
anchor=north,
legend columns=-1,
font=\scriptsize,
% text width=2.75in,
minimum height=0.20in,
nodes near coords style={rotate=90,  anchor=west, font=\tiny},
nodes near coords
},
]
\addplot[draw=black!100, fill=black!10] table[x index=0,y index=1] \dataset;
\addplot[draw=black!100, fill=black!30] table[x index=0,y index=2] \dataset;
\addplot[draw=black!100, fill=black!50] table[x index=0,y index=3] \dataset;
\end{axis}
\end{tikzpicture}
}
\caption{User preference among three implemented naming techniques (considering words in methods as terms)}
\label{fig:word}

\end{figure*}

\begin{figure*}[h!]
  \pgfplotstableread{
  

  1 3.5 3.133333333
  2 3.416666667 3.05
  3 3.45 3.216666667
}\dataset
\begin{tikzpicture}
\begin{axis}[xbar,
bar width=0.50cm,
% ylabel={\#Cluster no.},
% xlabel={Pyramid score},
yticklabel style={font=\footnotesize},
xticklabel style={font=\footnotesize, /pgf/number format/fixed},
xmin = 0,
xmax = 5.0,
yticklabels = {TFIDF, LDA, LSI},
%  symbolic y coords={TFIDF, LDA, LSI},
        ytick=data,
        legend entries={words\_in\_function\_names, function\_names},	
         nodes near coords, 
        nodes near coords align={horizontal},
legend style={
at={(0.5,-0.15)},
anchor=north,
legend columns=-1,
font=\scriptsize,
%text width=2.75in,
minimum height=0.20in}
]
    \addplot [draw=blue,
        pattern=horizontal lines ,
    ] table[x index=1,y index=0] \dataset;
    \addplot [draw=green,
        pattern=crosshatch ,
    ] table[x index=2,y index=0] \dataset;
\end{axis}
\end{tikzpicture}
  \caption{Comparison between three techniques considering function names and words in function names}~\label{fig:method_vs_word}
\end{figure*}

\subsection{User Rating on Words in Function Name Variant}
We have followed a similar approach to answer RQ3 that we used to answer RQ2. We averaged five participants' rankings for 12 clusters for the three techniques (TFIDF, LDA, LSI). In RQ3, we want to know participants' preference when we consider words in the function names as unit for the TFIDF, LDA, LSI-based techniques. In Figure \ref{fig:word}, we have plotted user rankings of the automatically suggested names for 12 clusters. Among twelve clusters, we can see that in seven of them, developers preferred names suggested by TFIDF and LSI technique in preference to the LDA technique, which covers almost 60\% of the clusters. Therefore, \texttt{RQ3} can be answered to establish that words in function name variation perform better with TFIDF, LSI than LDA.      


\subsection{Function Name vs. Words in Function Name}
For \texttt{RQ4}, we want to see users' preference on TFIDF, LDA, and LSI based techniques of the two variations we mentioned in \texttt{RQ2} and \texttt{RQ3}.
So, we averaged the user rankings of 12 clusters of three techniques from Figure \ref{fig:method} and Figure \ref{fig:word}. In Figure \ref{fig:method_vs_word}, we plotted the average ranks of the three techniques in two variations (i.e.,function\_names and  words\_in\_function\_names). From the figure, we can observe that developers preferred TFIDF, LDA, and LSI techniques with word as unit over method name variations. Words in function names variation get at least 5\% higher preference than the method names variations for each of the three techniques. 


\section{Threats to Validity}
We have used three subject systems for the user study, and all of them are written with the Python language. We acknowledge that our user sample size is small. To mitigate the effect of randomness, we used three different systems, considered four clusters from each of them and invited experienced developers for the study. Our approach depends on function names. Therefore, our approach would be less successful when the naming conventions are not properly followed. We have used open-source projects which generally maintain good naming conventions.  We have collected user summary after they evaluated six techniques to understand the limitation of automatic naming and provide feedback accordingly.  
% \vspace{6mm}
\section{Summary and Discussion}
\label{conclusion}
While proposing an approach to find concepts in source code from static call graph analysis, we try to remove the shortcomings of existing approaches in terms of techniques evaluation and use cases. We use two different variations of terms to recommend concepts that leverage developers' program perception effort while understanding a system. As program comprehension is a subjective matter, we collect user data to evaluate how our automatic labeling approach accords with user choice. The techniques we use are TFIDF, LDA, and LSI, with two variations (i. e., naming by function names, and naming by words in function names), where we found the TFIDF works better in cluster naming, and users prefer words in functions variants.   

During our manual analysis to generate a brief description of twelve clusters by observing execution paths, we found patterns in execution paths that might make the naming of concept cluster more human. In the next chapter, we will explore how we can generate more information for abstraction nodes.


\chapter{ Providing summary and significant patterns for abstraction nodes}
\label{chapter:hla2}
Program comprehension is a concept of understanding code to perform different software maintenance tasks. In recent years, the size of the code base is increasing drastically, which makes program comprehension difficult. To cope with the demand, many research works have been performed to understand how developers comprehend a program or code snippet and how to support developers to start their assigned tasks quickly. Different cognitive models are proposed in the literature to ease comprehension. 
Recently, studies to cluster execution paths of a call graph to aid overall program comprehension are going on.
We argue that this structure can be used to aid the top-down and bottom-up cognition model of program comprehension.
This paper has adopted new techniques for the hierarchical abstraction of a software system that performs better than the existing literature's related techniques. We conducted an exploratory case-study with three subject systems to validate our hypothesis. We found that it is possible to use this hierarchical presentation to enrich above mentioned cognition models.   

\section{Introduction}

One of the crucial parts of a software engineering job is software maintenance. Four types of software maintenance task are perfective, preventive, corrective, and adaptive \cite{williams2010characterizingArchitectureChanges}. To perform all of these tasks, developers first need to understand the target system, how its different components work together, and locate the relevant classes, methods, and files for completing a specific task. To add or change something in the system accurately and adequately, developers need to understand how its different components work together and map the implementation level source code to high-level features. Proper tool support for program comprehension can reduce the manual and economic cost of software maintenance, which will result in cheaper software \cite{arisholm2006impactUMLDocumentation}. In the literature, the studies on program comprehension are divided into two parts \cite{levy2019understandingLargeHierarchical}. First, how developers understand a code snippet. Second, understanding how large software systems are comprehended. Levy et al. \cite{levy2019understandingLargeHierarchical} conducted a study to find how comprehending a large system works from an experienced developer's perspective. The comprehension of a system has a conceptual and concrete level \cite{bass2003softwareArchitecturePractice, levy2019understandingLargeHierarchical}. In reverse software engineering, different tools are used to extract implementation level architecture from source code (call graph). Later, through manual analysis, they are mapped to concept level architecture, which helps cognitive mapping \cite{roy2008softwareArchitectureRecovery}. However, as software systems are getting more complex in size, manual analysis of implementation level architecture to high-level concepts requires more human resources. In most cases, they are exhausting. 

Studies on processing call graphs to facilitate overall system comprehension are very common in literature \cite{cornelissen2007understandingMassiveSequence, feng2018hierarchicalExecutionComprehension, reiss2005dynamicSoftwarePhases, watanabe2008featurePhaseDetection} . The dynamic call graph is used for most studies, which is good for specific test cases or scenarios. The problem with the dynamic call graph is they have redundancy problems and cannot capture the whole software systems \cite{gharibi2018automaticStaticCluster}. Recently research on overall system comprehension focused on static call graph took attention \cite{gharibi2018automaticStaticCluster, walunj2019graphevoEvolutionCall}. Execution paths from static call graphs \cite{pradel2009automaticUseageSpecification, salah2005scenariographerReverseEngineering} can be used to extract usage scenario or high level functionality. Clustering execution paths from both static and dynamic call graph pave the way for the hierarchical abstraction of the system \cite{feng2018hierarchicalExecutionComprehension, gharibi2018automaticStaticCluster}. Labeling these intermediate nodes of a cluster tree can create a hierarchical abstraction tree with concepts as the label. We argue that this type of tree representation can aid developers in using different program comprehension models. For example, the Bottom-up model is used by developers when they do not have any knowledge about the domain of the system. They gradually try to map low-level properties to high-level concepts. Developers can use the cluster tree of execution paths to facilitate Bottom-up cognition. The clustering starts from execution paths (low-level features) to a gradual grouping of similar paths, which are high-level features. Similarly, the cluster tree can help automate the top-down cognition model. 

In the top-down model, when developers have domain knowledge of a system, they try to map the knowledge to low-level implementations. The cluster tree hierarchically abstracts the features so that we have domain knowledge at the top of the tree that we can relate to low-level features by browsing the tree in a top-down manner. From our manual investigation to the proposed approach of Gharib et al. \cite{gharibi2018automaticStaticCluster}, we found that the abstraction tree has the potential to support program comprehension models automatically. However, they only used top-5 function names from the execution paths as the intermediate abstraction node label. We found that labeling the abstraction node properly with supporting documentation and example can make the hierarchical abstraction tree more attractive and comprehensive to the developers. This paper has proposed three techniques to improve the abstraction of intermediate nodes in a cluster tree. 

\begin{itemize}
    \item First, we experimented with labeling the nodes using TFIDF, LDA, and LSI information retrieval techniques. Previous studies only used the TFIDF technique.
    \item Second, we generated natural text descriptions for each node by summarizing comments from the execution paths' methods.
    \item Third, inspired by Feng et al. \cite{feng2018hierarchicalExecutionComprehension}, for each node, we attached significant patterns from execution paths by applying Sequential pattern mining. To validate our techniques, we conducted an exploratory case study with three subject systems to find how these techniques can automatically help developers in program comprehension. 
\end{itemize}

Our investigation shows that providing a natural text description and sample execution patterns increase the comprehensibility of abstraction nodes. 
  
The rest of our paper is organized as follows. Section \ref{approach} describes how the proposed approach works. In Section \ref{evaluation}, an exploratory case study is reported to validate our proposed techniques.


\section{Approach}
\label{approach}
\begin{figure*}[tb]
  \centering
  \includegraphics[width=\columnwidth]{figures/hla2/tree_structure.png}
  \caption{Structure of a hierarchical abstraction tree}~\label{fig:tree_structure}
\end{figure*}
In this section, we discuss two significant steps in our approach with a brief discussion. First, we described six steps to get the subject system's hierarchical abstraction tree in Section III-A. Second, in Section III-B, we describe how we used different information retrieval techniques to define the tree's hypothetical abstraction nodes. Data collection for evaluating the approach is depicted in algorithm \ref{alg:overall}.


\begin{algorithm}
    \SetKwInOut{Input}{Input}
    \SetKwInOut{Output}{Output}
    
    \underline{Call Graph to Concept Cluster Tree} $(call graph)$\;
    
    \Input{Call graph}
    \Output{Concept cluster tree}
    \For{Iterate each node in the call graph}
    {
        \If{ $Number\_of\_Incoming\_Degree(node) == 0$}
        {
            entryNodes.append(node);
        }
        \If{$Number\_of\_Outgoing\_Degree(node) == 0$}{
            exitNodes.append(node);
        }
    } 
    \For{$i\gets1$ \KwTo $entryNodes.length$ \KwBy $1$}
    {
        \For{$j\gets1$ \KwTo $exitNodes.length$ \KwBy $1$}
        {
            execution\_paths.append($simple\_DFS\_path(i, j)$)
        }
    }
    \For{$i\gets1$ \KwTo $execution\_paths.length$ \KwBy $1$}
    {
        \For{$j\gets1$ \KwTo $execution\_paths.length$ \KwBy $1$}
        {
            $distance\_matrix[i][j]$ = $consine\_similarity(i,j)$;
        }
    }
    $cluster\_tree$ = $create\_cluster\_tree(distance\_matrix)$;
    
    $hierarchical\_abstraction\_tree$ = $label\_clusters(cluster\_tree)$;
    
    return $hierarchical\_abstraction\_tree$;
    \caption{Our procedure for analyzing Python source code of a project to construct concept cluster tree}
    \label{alg:overall}
\end{algorithm}

% \vspace{4mm}
\subsection{Hierarchical abstraction tree of a software system}
A call graph is a visual representation of a software system's method invocation relationships between different methods. We adopted a static call graph, which is generated by analyzing source code. As a static call graph captures whole function calls of a target system, we choose to abstract the target system. Previous studies suggested that function names contain significant abstraction of source code. 
Thus, we emphasized mining keywords by analyzing function names in the static call graph.
As we wanted to abstract the whole system's high-level functionality hierarchically, therefore the decision to adopt the static call graph as a building-block of our approach is well-justified.

In Figure \ref{fig:tree_structure}, we presented the hierarchical abstraction tree structure. The leaf nodes of this tree are directly mapped to the execution paths. Execution paths are a list of function names executed sequentially during the execution of a software system. For instance, node 5 is mapped to the execution path where F11, F6, F7, and F9 are called sequentially. 
In this scenario, all the leaf nodes (4, 5, 6, 7) are mapped to four execution paths or function call sequences.
Node 1, 2, and 3 are hypothetical abstractions of the four leaf nodes.
Generating meaningful descriptions for these intermediate nodes can make the abstraction tree helpful towards different program comprehension cognition models.
In the figure, nodes 2, 3 have been labeled F11\_F6, F3\_F1, respectively. These labels are generated by analyzing their child nodes' function names. We plan to generate five keywords for each intermediate node, alongside a short natural text description (from the doc-string of function names) and few significant patterns from analyzing execution paths under investigation.

\begin{figure*}[tb]
  \centering
  \includegraphics[width=\columnwidth]{figures/hla2/approach_new.png}
  \caption{Overview of the overall approach}~\label{fig:overall}
\end{figure*}

\subsubsection{Analyzing source code using modified Pyan module}

We used a modified version of Python module Pyan \cite{pyan} to extract function relationships from a Python system. Pyan works only for a single directory. We adapted Pyan so that it can consider multiple directories while extracting the relationships. Pyan uses the AST tree for extracting function relationships. After analyzing the source code, we generated a graph in TGF format (Trivial Graph Format). In TGF format, all modules and functions' physical addresses in the source code are printed first. Then, relationships between all functions are presented as the caller and callee pair.

\subsubsection{Extracting function relationships from Trivial graph format}

Function relationships from the TGF file format are used as inputs in our technique. Encoded unique identifiers are used to replace function names for ease of processing during the hierarchical clustering step.

\subsubsection{Static call graph creation based on the extracted relationships}
To perform different graph operations, we have created graph objects of NetworkX \cite{networkx} module using the extracted function relationships. 
\subsubsection{Extracting execution paths}

The execution path is a simple path between an entry node and an exit node. An entry node is a node in the call graph which incoming edge degree is zero. So, no function is dependant on an entry node. An exit node is a node that has a zero degree of outgoing function calls. We have generated a list of entry and exit nodes to generate execution paths from a call graph. Simple Paths calculation algorithm of the NetworkX \cite{networkx}  is used in our approach, which applies the \texttt{DFS} algorithm to get the paths between a source node and a destination node. For our task, a source node is an entry node, and a destination node is an exit node.    

\subsubsection{Distance matrix for execution paths}

For clustering execution paths (sequence of function names), we need to measure the similarity between all pairs of execution paths. For this purpose, we  \cite{niwattanakul2013jaccardKeywordsSimilarity} implemented the Jaccard similarity measure. The linkage algorithm uses this similarity in the next step. If we have two sets $A$ and $B$, then their Jaccard similarity will be the ratio of the cardinality of their intersection by the union. As clustering algorithm works on distance in our case dissimilarity between two execution paths/clusters, we have subtracted the similarity score with one to get the dissimilarity value according to equation \ref{eq:jaccard}. After calculating dissimilarity between all pairs of execution paths, we converted the 2d matrix to 1d condensed matrix to make our program memory efficient.

\begin{equation}
Dis(A, B) = 1 - \frac{A\cap B}{A\cup B}
\label{eq:jaccard}
\end{equation}

\subsubsection{Clustering execution paths using linkage algorithms}

To group similar execution paths as clusters, we have implemented a linkage algorithm using the popular Python package Scipy \cite{scipy}. Scipy has a lot of different types of linkage algorithms already implemented in its core. To update the distance between two clusters, we have picked the Ward \cite{ward} minimum variance method. Equation \ref{eq:ward} \cite{scipy} shows how distance using the Ward method is updated when two clusters from cluster forest are merged into a new one.

\begin{equation}
     d(u, z) =   \sqrt{\frac{(n_x+n_z)d(x,z)^2+ (n_y+n_z)d(y,z)^2 - n_z d(x,y)^2 }{n_x+n_y+n_z}}
    \label{eq:ward}
\end{equation}

 
In equation \ref{eq:ward}, $u$ is a newly formed cluster, and $z$ is an unused cluster which will be used as reference to calculate distance. $n_x$, $n_y$ and $n_z$ are respectively the number of execution paths (as we are clustering the execution paths) in cluster $x$, $y$ and $z$.
When a new cluster $u$ is created, the distance between $u$ and all the other clusters are updated in the distance matrix. Additionally, cluster $x$ and $y$ are removed from the distance matrix as they have been merged as a new cluster $u$. This step is followed iteratively until only a single cluster remains in the cluster forest. 

% \vspace{4mm}
\subsection{Generating summary for the intermediate abstraction nodes}
After getting a tree by clustering execution paths in the previous step,  
we generate three types of summaries for each intermediate node. First, we used different 
information retrieval techniques like TFIDF, LDA, and LSI for selecting five keywords or five function names from analyzing execution paths descendant to an intermediate node. This information is the label of the intermediate nodes. Second, this time instead of considering the function names, we considered the function names' comments to provide natural text summary for each intermediate node. Comments from the functions are summarized using TextRank \cite{barrios2016variationsTextRankSummarization} algorithm. Given a collection of sentences as input, this algorithm can summarize the collection to a fixed number of sentences. Third, inspired by Feng et al. \cite{feng2018hierarchicalExecutionComprehension}, to provide a glimpse of the significant patterns among execution paths SPAM (sequential pattern mining) algorithm PrefixSpan \cite{han2001prefixspanSequentialPatterns} is implemented. We find that all the execution paths in an intermediate node share some patterns from our manual investigation of the execution paths. By taking a look at the significant patterns, we can comprehend more elaborately about the intermediate nodes. We present these patterns in support of the Label and Summary generated by the previous two steps. Therefore, to comprehend an abstraction node, we have a label, summary description, and patterns from the execution paths. Below we briefly described the steps. 

\subsubsection{Label intermediate/abstraction nodes}
\begin{itemize}
    \item TFIDF:
TFIDF \cite{ramos2003usingTfidfRelevance} is a very popular information retrieval technique widely used in text-based search engines. The full form of the TFIDF is term frequency-inverse document frequency. Term frequency means how frequent a term in a document is. Term frequency is calculated according to equation \ref{eq:tf}.

\begin{equation}
    tf = f_{t,d}
    \label{eq:tf}
\end{equation}
\begin{equation}
    idf = \log(\frac{n}{df(t)})+1
    \label{eq:idf}
\end{equation}
\begin{equation}
    tf-idf(t,d) = tf(t,d) * idf(t)
    \label{eq:TFIDF}
\end{equation}
The function $f_{t,d}$ counts frequency of term $t$ in the document $d$. Inverse document frequency is calculated according to equation \ref{eq:idf}. Function $df(t)$ in equation \ref{eq:idf} is the count of documents term $t$ is present. The main purpose of $idf$ is to penalize common keywords in the corpus. Term frequency (tf) and Inverse document frequencies (idf) are multiplied to get score for terms. We have adopted \texttt{TFIDFVectorizer} class of scikit-learn \cite{scikit-learn} library for implementing $TFIDF$ technique.
    \item LDA:
Latent Dirichlet Allocation (LDA) \cite{blei2003latentLDA} is a statistical model that tries to describe a set of documents by assuming they are created from some topics. LDA is a popular topic modeling technique. LDA assumes every term in a document belongs to some topic. So, it considers each term belongs to some topic and then performs analysis to find which assumptions are supported by statistics of the corpus. We have used Gensim \cite{gensim} library for implementing LDA for our approach.
    \item LSI:
Latent Semantic Indexing (LSI) \cite{deerwester1990indexingLSI} is a technique used in natural language processing. LSI assumes semantically similar words occur together. First, the term-document frequency matrix is calculated from the corpus. This term-document frequency matrix is decomposed into three matrices using the Single Value Decomposition (SVD) technique. Terms are first assigned to topics using the term-document frequency matrix. Then, using all the topics, a topic importance matrix is derived, which leads to topics for the documents. Similar to LDA, we used Gensim \cite{gensim} library for implementing LSI. 
\end{itemize}

\subsubsection{Summary description of intermediate/abstraction node}
To generate a summary for node 3 of Figure \ref{fig:tree_structure}, we collect the first line of docstring comment for the function F1, F2, F3, F4, F6, F9 as they consist of the execution paths of node 3's descendant nodes. Next, we remove duplicates from the comments and provide these sentences to TextRank \cite{barrios2016variationsTextRankSummarization} algorithm to generate summary. There are many functions in an execution path for real-world software, so using the TextRank algorithm, we get a short five sentence comprehensive summary. 

TextRank~\cite{barrios2016variationsTextRankSummarization} is a graph-based automatic summarization technique. TextRank is language and domain-independent. To generate a summary, training a corpus is not required, making it suitable for our task. All the sentences of the target document make the nodes of a graph. Edges between the nodes are created using different similarity measures between two nodes or sentences. At last, PageRank~\cite{page1999pagerank} algorithm is used to obtain a summary from the graph.

\subsubsection{Significant patterns for intermediate/abstraction node}
To get significant patterns for node 3 in Figure \ref{fig:tree_structure}, we have to analyze execution paths of node 6 and node 7. The execution paths of node 6 and 7 have $ F3 \rightarrow  F1 $ sequence common. So, presenting this common sequence as a significant pattern for node 3 make a good abstraction of descendant execution paths of node 3. To mine this sequential patterns, we implement PrefixSpan \cite{han2001prefixspanSequentialPatterns} sequential pattern mining algorithm. If we provide a collection of execution paths to PrefixSpan, it gives a significant pattern analyzing the collections. PrefixSpan creates a prefixed based projection database to find sequential patterns efficiently.

\section{Experimental Design}
\label{evaluation}
This section will discuss research questions that drive this study, how we collected our subject systems, what criteria were considered, and how we designed our exploratory case study. 

\subsection{Research Questions}
In this paper, we tried to improve the comprehensiveness of the abstraction of nodes. First, we split function names to get words in the names so that TFIDF, LDA, and LSI methods perform more naturally. There is also another benefit of using words in method names as they will be fixed length. We investigate how effective node names using the word variant in our RQ1. Besides, we attach a natural text summary for each node using the docstring of functions, which consist of RQ2. Similarly, we generate significant patterns from execution paths to support node comprehension, and this is our RQ3. Finally, we investigate how merging the results of RQ1, RQ2, and RQ3 improve the abstraction tree in RQ4. 

\begin{itemize}
    
    \item \textbf{\texttt{RQ1}} How effective is word variation of TFIDF compared to method variation?
    \item \textbf{\texttt{RQ2}} How comprehensive is natural text summary for abstraction nodes?
    \item \textbf{\texttt{RQ3}} How effective are the mined patterns to comprehend abstraction nodes?
    \item \textbf{\texttt{RQ4}} How effective comprehension of an abstraction node, if label, summary, and patterns used together?
\end{itemize}

\subsection{Dataset Collection}
In this study, we have experimented with three subject systems. Source code of the subject systems are cloned from their Github repository. Pyan library is used to extract caller-callee relationships in trivial graph format (TGF). Next, we created a networkX graph object to iterate through the call graph and extract execution paths. At last, the ward linkage clustering algorithm is used to create a hierarchical abstraction tree. In table \ref{table:subject_systems}, we present the entry, exit nodes, line of code, number of execution paths. We choose our subject systems carefully to have three different sizes of execution paths as the number of execution paths determines how big the abstraction tree will be. We wanted to keep the size flexible for performing our analysis to find our proposed techniques' effectiveness. 

\subsection{Case study design}
To find the effectiveness of the proposed techniques, we carefully choose different abstraction nodes and their neighborhood. After that, we manually checked whether the label, summary, and mined patterns properly abstract and describe the system's high-level concepts. To verify whether the approaches properly support our claim, we manually browsed the source code of target systems to know the systems' high-level concepts. To minimize subjective bias, two of the co-authors of this paper differently analyzed the selected abstraction nodes and their neighbors. To generalize our findings to some extent, we have used three different subject systems so that our claim is acceptable.

\begin{table*}% put at top of page if possible 
 \caption{3 Subject Systems with their No. Entry, Exit Nodes, LOC, Paths, And Date Retrieved}
\centering
% \resizebox{3.4in}{!}{
\begin{tabular}{l|l|l|l|l|l|l|l}
% (2700, 0) (1080,1) (79400,2) (1880,3) (1790,4) (2600,5) (2600,6) (19000,7) (903,8) (1320,9) https://github.com/flutter/flutter https://github.com/facebook/react-native
No & URL & Name & Entry Nodes & Exit Nodes & LOC & Paths & Date retrieved\\
 & (https://github.com) &  &  &   & & &\\
\hline
1 & \url{Our code}& higher\_level\_abstraction & 2 & 22 & 999 & 31 &  28 May 2020\\
2 & \url{/davidfraser/pyan} & pyan & 36 & 50 & 3711 & 637 & 28 May 2020\\
3 & \url{/CorentinJ/Real-Time-Voice-Cloning}& Real-Time-Voice-Cloning & 21 & 93 & 9117 & 404 & 28 May 2020\\


\end{tabular}
% }
\label{table:subject_systems}
\end{table*}

\section{An exploratory case-study}
\begin{figure*}[tb]
  \centering
  \includegraphics[width=1\columnwidth]{figures/hla2/rq1_hla1.png}
  \caption{Snippet from subject system 1 (Our code)}~\label{fig:rq1_hla1}
\end{figure*}

\subsection{ RQ1: Effectiveness of word variation labeling}
To see the effectiveness of labeling, we manually picked the root node and its neighborhood. We explored a similar tree snippet for the three systems. In Figure \ref{fig:rq1_hla1}, we see root node 60 has the label \textit{lda pair get docstring jaccard}. From this label, one can guess that something related to docstring, jaccard distance, and topic modeling LDA occurs in the higher\_level\_abstraction subject system. An interesting thing to notice is that name of node 60 and 59 is the same. Although node 58, which is a child node of 60 has two new keywords py and view that indicate something related to Python file and view occurs inside the nodes' execution paths. On the other hand, if we see the name for node 60 using TFIDF method variant( \textit{pretty\_print\_leaf\_node bfs\_with\_parent mining\_sequential\_patterns id\_to\_sentence cluster\_view}) , we see that using method as unit for TFIDF is more comprehensible than using word as unit for TFIDF. Another benefit of TFIDF method variant is for node 60 and 59, it provides different names according to their execution paths. On the other side, the word variant of TFIDF gives the same name for node 60 and 59 because of over\-fitting. 

\begin{figure*}[tb]
  \centering
  \includegraphics[width=\columnwidth]{figures/hla2/rq1_pyan1.png}
  \caption{Snippet from subject system 2 (pyan)}~\label{fig:rq1_pyan1}
\end{figure*}

In figure \ref{fig:rq1_pyan1} shows that we have a snippet of the Pyan subject system's abstraction tree. Pyan~\cite{pyan} is an open-source software which can extract call graph from a Python project. From our general knowledge, we can expect the concepts related to source code. If we look at node 1272 at Figure \ref{fig:rq1_pyan1}, the name is \textit{c3 module label use idx}. Except for the module, other keywords are not that much expressive. For node 1268, we see keywords like class, node, namespace indicate that the node is relevant to processing source code. However, we can see a recurrent occurrence of the same name for node 1272, 1271, which is an over-fit situation. Name for node 1272, 1271 using method variant TFIDF are 
\textit{write\_edge  find\_filenames  DotWriter}, \textit{write\_edge TgfWriter DotWriter visit\_Assign}  which clearly indicates some hint what the nodes do. 

In Figure \ref{fig:rq1_realTime1}, we have a snippet from our third subject system (Real-Time-Voice-Cloning \cite{realTime}). This open-source project can clone someone's voice from a clip of at least five seconds. So, this system's high-level functionalities can be converting wave length, processing audio, training model. The name of root node 806 is \textit{synthesize train synthesizer synthesize toolbox}. Here, train indicates training models, synthesize means processing audio signal, and toolbox indicates the tool system. For node 791, we see keywords like encoder, spec which indicates processing of signals. Using method name variant of TFIDF the name for node 806 and 791 are \textit{save\_wav encoder.audio discretized\_mix\_logistic\_loss profile\_noise encoder.visualizations}, \textit{current\_encoder\_fpath make\_spectrogram load\_preprocess\_wav normalize\_volume}. TFIDF method variant provides more contextual information from the name of node 806, 791. 

From the above manual investigation of node names using the method and word variant, it is evident that using method name variant provides more semantic abstraction. However, word variant provides a fixed-length  name. Yet, the output is ambiguous. So, We would recommend using method variant TFIDF to name or label an abstractio node.


\begin{figure*}[tb]
  \centering
  \includegraphics[width=\columnwidth]{figures/hla2/rq1_realTime1.png}
  \caption{Snippet from subject system 3 (Real-Time-Voice-Cloning)}~\label{fig:rq1_realTime1}
\end{figure*}



\subsection{ RQ2: Natural text summary for abstraction nodes}
 Natural text is more comprehensive than a few keywords. Therefore, to support abstraction nodes' comprehension in a hierarchical tree, we propose to summarize the methods' docstring in all execution paths of the node. To keep the importance of each method, we used only the first line of the docstring. Also, from our manual analysis, it is evident that most of the time, the first line describes the function's purpose. Although for many cases, we found that docstring is absent. In those cases, we just omitted the method for generating summary. To answer our RQ2, we investigated the summary for nodes in three subject systems. 
 
 The root node 60 of subject system 1 has the text summary \textit{clustering execution paths using scipy Labeling a cluster using six variants  This function returns function name with their docstring  analyzing Python programs to build cluster tree of execution paths.} Subject system 1 is our program to cluster execution paths from a call graph. Then, we labeled the nodes in cluster using six different techniques and also analyzed docstring to produce summary that we discussed in this research question. If we carefully observe the summary for node 60, using TextRank algorithm our produced summary is almost accurate about what the first subject system does. For node 57, our approach's summary is \textit{converting tgf file to a networkX graph extracting function names from TGF file analyzing Python programs to build cluster tree of execution paths.} From the summary, we can confidently tell that abstraction node 57 deals with extracting function names from TGF file, converting TGF format file to networkX graph.
 
 
The root node 1272 of subject system 2 (Pyan) has the summary \textit{Resolve those calls to built-in functions whose return values Return a label for this node, suitable for use in graph formats.} As Pyan deals with source code, we can see the summary is saying something about resolving built-in functions, labeling nodes for graph format. We can relate this summary to the purpose of Pyan partially. For node 1271, the summary is \textit{Try to determine the full module name of a source file, by figuring out       Return the node representing the current class, or None if not inside a class definition.}
The summary for node 1271 says that the execution paths it abstracted mostly deal with determining a source file module, getting the class name a node represents. These are some standard utilities for a project which process source code. The summary for node 58 is \textit{Generate cluster figure from a dendrogram. Flattens a nested list. This function returns function name with their docstring.} Node 58 deals with plotting the dendrogram, mapping function name to docstring.

The root node 806 of subject system 3 (Real-Time-Voice-Cloning) has the summary \textit{If this function is not explicitely called, it will be run on the Args:                  Computes where to split an utterance waveform and its corresponding mel spectrogram to obtain   Derives a mel spectrogram ready to be used by the encoder from a preprocessed audio waveform.} As we have described previously, Real-Time-Voice-Cloning software can clone a voice to produce speech from text. If we see the summary generated by TextRank for node 806, we can say it deals with processing audio wave-forms. Furthermore, for node 801, the summary is \textit{Args:   Synthesizes mel spectrograms from texts and speaker embeddings.} Summary for node 801 is very small. It indicates that mostly docstring for Real-Time-Voice-Cloning is empty, and the short summary indicates text to speaker embedding, which is essential for voice cloning.

From the observation of node summary generated by TextRank for three subject systems, we can conclude that if functions are properly documented with docstring this approach can complement the comprehensiveness of abstraction nodes. We faced the challenge of different formats of comments, which hampered the extraction of the docstring. 

\subsection{ RQ3: Effectiveness of mined patterns from execution paths}
From our manual investigation into the execution paths of an abstracted node, we find that there are recurrent patterns that can help comprehend the abstracted node. Therefore, we develop a technique to use sequential pattern mining for selecting patterns among the execution paths from those findings. 

The patterns for root node 60 of subject system 1 are

\begin{itemize}
    \item ClusteringCallGraph,python\_analysis,clustering\_using\_\\
    scipy
    \item ClusteringCallGraph,python\_analysis,clustering\_using\\
    \_scipy ,labeling\_cluster
    \item ClusteringCallGraph,python\_analysis,clustering\_using\_\\scipy,
labeling\_cluster, tf\_idf\_score\_for\_scipy\_cluster
\end{itemize}
From observing this pattern, we can clearly tell that node 60 is working with Python code, clustering using scipy library, labeling the clusters. As this is the root node of the subject system 1, we can conclude the patterns clearly represent the root nodes purpose.

The patterns for node 58 are 

\begin{itemize}
    \item ClusteringCallGraph,PlayingWithAST
    \item ClusteringCallGraph,get\_all\_method\_docstring\_pair\_of\_a\_\\
    project
    \item ClusteringCallGraph,get\_all\_method\_docstring\_pair\_of\_a\_\\
    project,get\_all\_py\_files
\end{itemize}
From the patterns for node 58 retrieved by sequential pattern mining, we can see it is extracting docstring from all Python files which is one of the important part for answering our RQ2. 

The patterns for root node 1272 are

\begin{itemize}
    \item get\_attribute
    \item resolve\_builtins,get\_attribute
    \item analyze\_binding,resolve\_builtins
\end{itemize}

From the list of patterns, we can see there is very little information. Although these patterns are for the root node, they are most frequent. Limiting the length of the minimum pattern can solve the problem. However, we can understand that getting attributes, analyzing bindings, and resolving built-ins is the most common concept for root node 1272.

The patterns for node 1240 are 
\begin{itemize}
    \item resolve\_builtins,resolve\_method\_resolution\_order,\\
    C3\_linearize,C3\_merge
    \item analyze\_binding,resolve\_builtins,resolve\_\\method\_resolution\_
    order,C3
    \_linearize,C3\_merge
    \item resolve\_builtins,resolve\_method\_resolution\_order,\\
    C3\_linearize,C3\_merge,C3\_find\_good\_head,\\LinearizationImpossible
\end{itemize}
From the patterns of node 1240, we can see that method resolution order, linearize, resolve builtins are the main task.

The patterns for root node 806 of subject system 3 are

\begin{itemize}
    \item init,setup\_events
    \item wav\_to\_mel\_spectrogram
    \item embed\_utterance
    \item train
\end{itemize}

From the patterns for node 806, we see that it is creating different events, converting wave to spectrogram, and training model, which summarizes what Real\-Time\-Voice\-Cloning does. 
The patterns for node 804 are

\begin{itemize}
    \item wav\_to\_mel\_spectrogram
    \item encoder\_preprocess
    \item embed\_utterance
    \item encoder\_preprocess,\_preprocess\_speaker\_dirs,\\
    preprocess\_speaker
\end{itemize}
From the patterns of node 804, we can say that node 804 is embeddeing and encoding audio signals, prepossessing speaker audios.

From observing significant patterns of different nodes from three subject systems, we can conclude that providing them with an abstraction node can enhance a node's comprehensibility. However, the minimum length of each pattern and removing frequency-based bias should be considered to improve the patterns.
\subsection{ RQ4: Effectiveness of using label, summary and patterns together}

In RQ1, we manually analyzed how expressive the label for nodes using word and method variants. We found that method variation of the TFIDF technique provides a more sophisticated label than its word variant, which seems ambiguous. From our analysis of RQ2, we have seen promising summary for nodes using TextRank. Although this method's success hugely depends on how well the method docstring is written, excluding unrelated information is a challenge due to different formations. From RQ3, it is clear that patterns from execution paths are helpful to support nodes, although effectiveness hugely depends on selecting tuning mining pattern algorithms. Therefore, if the challenges for generating name, summary, and patterns are solved accordingly, they will enrich the comprehension of the abstraction node, in total, the overall hierarchical abstraction tree. 

\section{Threats to validity}

We have picked three different subject systems of varying size so that our approach's effectiveness can be generalized to some extent. We manually analyzed the results of our techniques to reach a saturated decision. Furthermore, two of the authors of this paper individually analyzed the findings to remove subjective biases. We carefully picked the first line skipping lines with special characters to extract the docstring for each method. 

\section{Conclusion and Future Plan}
In software engineering, program comprehension is an important research area that involves many other software maintenance tasks. Nowadays, the size and complexity are growing. To perform a maintenance task, developers need to understand how different components of the system interact. Other cognition models are studied in the literature to aid developers. Top-down and bottom-up models are popular program comprehension models. In these models, developers map high-level features with low-level implementations depending on a specific situation. Different hierarchical abstraction techniques which use call graph of dynamic and static variation exists. 

This study focused on improving a software system's abstraction hierarchically using execution paths from a static call graph. Executions paths represent low-level implementation. Grouping execution paths in a cluster tree, a software system is hierarchically abstracted. Information presented with the nodes of a cluster tree is useful for developers to map high-level features to low-level implementations. We proposed different techniques like using word and method variant for TFIDF to label nodes, generated summary for each node from method docstring, and mined significant patterns to attach all these three types of information with each node to aid comprehension.

To evaluate our approach, we conducted an exploratory case study to determine our proposed techniques' effectiveness. We discussed the generated output for different nodes and challenges to improve. We found that generalizing the techniques with more subject system would improve the techniques. In the future, we plan to use source code summarizing techniques to produce more accurate summary. Moreover, we plan to build an automated tool that, given a software system (Python), will produce a hierarchical abstraction tree that developers can browse interactively. We have a plan to conduct a wide-scale user study to evaluate these techniques.


\chapter{Finding effectiveness of abstract code summary tree}
In this chapter, we introduce different clustering technique in Section \ref{hla3:approach}. Next, we present a  human-subject study of the developed HCPC tool in Section \ref{hla3:human_study}. In Section \ref{hla3:interface} and \ref{hla3:implementation}, we discuss the interface and implementation of the HCPC tool. Last, in Section \ref{hla3:use_guide} and \ref{hla3:case_study}, we discuss how to use HCPC tool for two use cases and present an example using \emph{jupyter\_client} project.

\section{Motivation}
Finding relevant methods, classes and files is frequent part of daily activities of a software developer. Most of the software maintenance tasks require to find relevant locations for solving the task. As the size of codebase grows, it becomes difficult to remember everything in detail. Therefore, common practice is to figure out some relevant keywords and search for the files containing the keywords. The problem with this approach is search results are random and it gives no idea of exploring the codebase according to the order different components are called.

In the previous studies, we have advanced the existing works on hierarchical abstraction of static execution paths by finding appropriate technique to label noes in the tree and further complement the nodes with natural text summary and execution patterns for better comprehension. In this study, our motivation is to make the abstraction tree usable for developers daily concept location activities. We have found two scope of improvement from the previous studies. First, we observe the abstraction tree became complex to explore as the number of execution path grows. Therefore, we implemented a cluster flattening technique to have more flexibility and simple structure with cut-off depth. Second, we have changed the similarity metric for comparing execution paths from Jaccard distance to match\_a\_strike score. By updating the similarity measure, we ensure more accurate grouping of clusters. After the technique changes, we have developed a tool called HCPC for doing a human-subject study to find the effectiveness of HCPC. In the HCPC tool, we added a node highlight feature where specific function can be selected to highlight relevant nodes. From our study with developers, we have found that HCPC tool can be helpful for exploring codebase in a guided way in daily software maintenance activities. 



\section{Approach}
\label{hla3:approach}
In this study, we have followed similar steps as study 1 and 2 except two changes. First, we have changed the similarity score from Jaccard distance to strike\_a\_match algorithm. Strike\_a\_match algorithm takes into account the contents of two list and the sequence they appear. On the contrary, Jaccard distance only considers the content of two list. To improve the clustering result, we have made the change in similarity metrics. Second, we have added one more step to reach the final abstract code summary tree. Previously we have used the step by step tree returned by linkage algorithm. However, the linkage tree is not flexible for browsing. Therefore, we have used cluster flattening technique to get more flexible 5-6 depth tree. We discuss strike\_a\_match, node summary, execution pattern and cluster flattening technique in the below subsections. 

\subsection{strike\_a\_match}
In Algorithm \ref{alg:strike_a_match}, we provided the pseudo code for reproducing strike\_a\_match algorithm\footnote{http://www.catalysoft.com/articles/strikeamatch.html}. The methods takes input two list which is execution path one and two. The method returns a similarity score between 0 and 1 where 0 means no match and 1 means full match. In line 2-3, all method pairs in consecutive order are generated. In line 4, we calculate union value by summing length of the two generated pair lists. From line 6 to 14, we iterate over the pair lists and see if they match to calculate intersection value. When we find a match, we remove the pair from $ep2\_pairs$ to avoid considering the same match again. At last, we return the similarity score using union and intersection value. The method considers the order in addition to the content of two lists. 


\begin{algorithm}
    \SetKwInOut{Input}{Input}
    \SetKwInOut{Output}{Output}
    
    \underline{Compare\_execution\_paths} 
    
    \Input{ep1, ep2}
    \Output{similarity\_score}
    \tcp{method\_pairs returns all the two length consecutive pairs from execution paths}
    ep1\_pairs = method\_pairs(ep1)\; 
    ep2\_pairs = method\_pairs(ep2)\;
    union = len(ep1\_pairs) + len(ep2\_pairs)\;
    intersection = 0\;
    
    \For{$i\gets0$ \KwTo $len(ep1\_pairs)$ }{
        \For{$j\gets0$ \KwTo $len(ep2\_pairs)$ }{
            
            \If{ep1\_pairs[i] == ep2\_pairs[j]}{
                intersection += 1 \;
                ep2\_pairs.pop(j)\;
                \textbf{break}\;
            }
        }
    }
    \textbf{return} ( 2 * intersection) / union
    
    \caption{Strike\_A\_Match algorithm}
    \label{alg:strike_a_match}
\end{algorithm}

\subsection{Node summary}

In Algorithm \ref{alg:node_summary}, we provided the pseudo code for generating node summary for each abstraction node. The two input of the method are execution paths and function\_id to comment dictionary. The execution paths is a 2D list where each row corresponds to a execution path and cells contain function id. The second argument is a dictionary where function id are mapped to their first line docstring comment. In line 3 - 7, we iterate through all the execution paths and all functions in a execution path. We add all the comments to $all\_comments$ variable for use in summarize. In line 8, we provoke the summarize method from Gensim~\cite{gensim} library which by default returns one-third of the $all\_comments$ as summary using  TextRank~\cite{mihalcea2004textrank} algorithm. We have tried with different ratio of input to summary ratio and found the default settings sufficient for our purpose.

\begin{algorithm}
    \SetKwInOut{Input}{Input}
    \SetKwInOut{Output}{Output}

    \underline{Generate\_node\_summary} 
    
    \Input{execution\_paths, function\_id\_to\_comment}
    \Output{node\_summary}
    all\_comments = ` '\;
    \For{execution\_path \textbf{in} execution\_paths}
    {
        \For{function\_id \textbf{in} execution\_path}
        {
            all\_comments += function\_id\_to\_comment[function\_id]\;
        }
    }
    
    node\_summary = summarize(all\_comments)\; \tcp{summarize by Gensim}
    \textbf{return} node\_summary
    \caption{Generate node summary from execution paths of an abstraction node}
    \label{alg:node_summary}
\end{algorithm}

\subsection{Execution patterns}
In Algorithm \ref{alg:execution_patterns}, we present pseudo code for generating execution patterns. The method takes execution paths as input and outputs frequent patterns found by analyzing all the execution paths. For our approach, we have mined top-15 frequent patterns. Execution paths is a list of list where the inside list is list of function ids which can be called sequentially. We use the PrefixSpan algorithm which mines frequent pattern from a set of lists. We use the topk method to get top-15 execution patterns. We use default settings for maximum length and minimum length of the patterns. 

\begin{algorithm}
    \SetKwInOut{Input}{Input}
    \SetKwInOut{Output}{Output}
    
    \underline{Generate\_Execution\_Patterns} 
    
    \Input{execution\_paths}
    \Output{execution\_patterns}
    NUMBER\_OF\_PATTERNS = 15\;
    ps = PrefixSpan(execution\_paths)\;
    top\_patterns = ps.topk(NUMBER\_OF\_PATTERNS)\; 
    
    \textbf{return} top\_patterns\;
    \caption{Generate node summary from execution paths of an abstraction node}
    \label{alg:execution_patterns}
\end{algorithm}


\subsection{Cluster flatten technique}

In previous studies, we have used step-by-step clustering tree as abstract code summary tree. However, for $n$ number of execution paths, the abstraction tree will have $2n - 1$ nodes which is not practical for medium to large projects. Therefore, we processed the cluster tree by using cluster flattening. We flatten the cluster tree first by a larger distance which will return very few number of clusters. Next, we decrease the distance which increases the number of unique clusters. We continue the process for having five steps of clustering. We also memorize the previous clusters to link the abstraction tree step-by-step.


\section{Human-subject Study}
\label{hla3:human_study}
To evaluate the effectiveness of HCPC, we contacted with \emph{scidatamanager} development team. We have collected their source code to analyze by our system. We have invited the developers of the \emph{scidatamanager} project to validate whether HCPC can be helpful for getting overview of the \emph{scidatamanager} and accomplish a specific software maintenance task.

\subsection{Research Questions}
\label{hla3:evaluation}
We want to evaluate the effectiveness of HCPC for helping developers comprehend a software project. We address two research questions:
\begin{itemize}
    \item \textbf{RQ1:} To what extent developers agree with our approach for getting overview of a project?
    \item \textbf{RQ2:} How helpful our approach to understand relevant high-level concepts targeting a low-level source code?
\end{itemize}
\subsection{Study Design}
The interview with developers are conducted remotely via Skype. The interview process was divided into four steps:
\begin{itemize}
    \item \emph{Introduction:} First, we brief each participants about our research. Then, we share our screen to show how to use HCPC tool. We demonstrate HCPC tool by exploring \emph{jupyter-client} project. We also discuss different components role to help program comprehension. Later, we asked the participants to go to a specific URL where our application is hosted and share their screen. We informed participants about two parts of the study.   
    \item \emph{Feedback on getting overview (RQ1):} In this phase, we asked the participants to explore the Cluster tree alongside different components like Node summary, Execution patterns. We requested them to check whether they can get overview of the \emph{scidatamanager} project. We encouraged the participants to express their thought while they explore different parts of the system. At the same time, we observe the participants interaction with the system and noted any feedback provided by them. When they explored the tree, we asked them whether the keywords and groups provide any reasonable clue about what the system does. Similarly, we asked them about their opinion on Node summary and Execution Patterns. We also inquired whether they have any suggestion or expectation on the components to be more helpful.
    
    
    \item \emph{Effectiveness of finding help for specific task (RQ2):} After we complete the second step, we move on to the third phase. In this step, we asked the participants to use the search option to find relevant nodes in the cluster tree and see whether they can find any help to do any specific task. We have encouraged them to remember any recent feature or issue they solved and try to see whether the HCPC tool could help them for completing the tasks. We asked the participants about how helpful Node summary, Execution patterns and the highlight of execution paths can be for someone new to the codebase to accomplish the tasks.
    \item \emph{Open discussion and closing:} At the end, we asked some open-ended questions about any suggestion and future features. The meetings lasted between 40 to 60 minutes. We ended the meeting thanking the participants for their valuable feedback and time.

\end{itemize}
\subsection{Participants and Subject System Selection}
While observing the HCPC tool output for \emph{jupyter-client} project, we can relate the different nodes content to the components in \emph{jupyter-client} documentation. We decided to conduct the study on a subject system where the team members can participate in the study to evaluate HCPC tool performance on their known codebase. We contacted the \emph{scidatamanager} team whether they could share their source code and participate in the study to evaluate HCPC. The development team of the \emph{scidatamanager} project agreed to share the codebase and participate in the study. 
\subsection{Results}
\textbf{Answering RQ1.} Participants mostly agreed that the HCPC tool can help getting overview of their project. When we asked the participants, they started to explore the abstraction tree by carefully observing the keywords for each node and expanding to child nodes. The participants agreed that high-level nodes provide hint to the features in their project. For example, participant P3 said, \emph{``I can relate to different basic components from high level nodes. If someone new joins the team, they can start from top nodes and see the path patterns for getting most frequent behaviour and then explore the code-base easily."} Participants appreciated the node summary as it states in plain text what are the purposes of the keywords in the project. Participants also find that when they see node summary for deeper nodes, the summary becomes more precise for specific features. 
According to participant P1, \emph{`` This part is helpful as it states in natural texts instead of a few words. Another interesting fact about the summary is when going deeper the summary became more precise.''} While exploring the execution patterns, we observed that participants find it helpful to know some frequent call sequences in specific nodes. However, participant P2, P3 suggested that having the frequency with the patterns would be interesting to know for understanding the importance. 

In summary, \textbf{Participants find HCPC tool helpful for getting overview of their software system with node title, summary and execution patterns.} According to their final feedback for comprehending overview, they pointed out that HCPC tool has the potential to decrease get started time for a project. According to participant P1, they believe  it can help to decrease getting started time around 50\%-60\%.

\textbf{Answering RQ2.} Participants find it useful to be able to search for specific keywords. From the interview, we observed that developers tried to highlight nodes for some recent work they have done or something they are familiar with to check how the HCPC is representing the relevant concepts. For example, participant P3 tried to highlight the nodes related to dataset publishing as it is one of the core feature of the project. While browsing the highlighted nodes and its supporting contents (node summary, execution patterns, execution paths), participant P3 identified that it is possible to know similar paths where the function is called. Another interesting observation by participant P2 is, \emph{`` I see the nodes can be searched by functions. In addition, I would love to see filters such as class, files.''} Participant P3 shared from their previous experience that sometimes they have to fix some issues of another project which not very well documented and they struggle a lot to figure out the abstraction patterns followed in the codebase. Both participant P3, P1 suggested using the search option to explore execution paths will be helpful to decrease time required for completing tasks in those scenarios. Another interesting observation from the interview is for some searches multiple nodes are highlighted which shows the specific functions being used in different scenarios. We observed participants was enthusiastic to know what are the different directions the function is being used by going deeper in the abstraction tree. In addition, participant P1 shared that many times they try to search the codebase with some keywords using the find option provided by the editor to retrieve relevant files. However, the search result does not show any order or how these class or methods are being called. They suggested that with the execution patterns and paths HCPC can help replacing the raw find workflow into more execution based search process. In summary, \textbf{the feedback from the participants and our observation during the interview it is viable that the search option of  HCPC has the potential to help in day-to-day software maintenance activities. } 

% To Do using find button for searching which do not have any order

During our open-ended questions and suggestions, we found valuable feedback for future development and adaptation of the HCPC tool. One important suggestion is to incorporate automatic comment generation techniques for methods which have no comments. This will be a valuable future work suggestion for our HCPC tool, as it will be helpful for projects which do not follow best practices. Another worth mentioning future work is suggested by participant P2 is to generate report of the abstraction structure where developers can edit the components name according to their understanding from the HCPC tool. These report can be used as a documentation of the project structure from static execution perspective. In addition, participants suggested to enable the option to export projects from GitHub which will be useful for quickly exploring a new codebase. From the above discussion, \textbf{we can conclude that HCPC tool can help to get overview of a software project from static execution perspective and can be used to help doing a specific task in hand.}


% TODO
% github, report generation 


\subsection{Threats to validity}
To address external validity, we have collected a software project which is developed in industry settings instead of working with a sample project. We have selected a industry project to ensure generability to some extent. Although the subject system is written in Python, our approach will work with other static typed language.

To address internal validity, we have tried to minimize any communication issue by repeating the feedback when in doubt. We asked open-ended questions at the end so that participants are able to provide feedback outside the questions asked. 
% To Do communication, subject generability industry project, language, open ended question


\section{Implementation }
\label{hla3:implementation}
In this section, we briefly highlight different parts of our implementation as shown in Figure \ref{fig:architecture}.

\begin{enumerate}
    \item We clone the source code from GitHub in a temporary folder. The source code will be used in the next phase by Python static code analyzer.
    \item We use Pyan~\cite{pyan} as static Python code analyzer. Pyan goes through all the \emph{*.py} files looking for which method calls which method. Pyan generates a text file  which encodes all the methods with numbers and then contains which method calls which method. We generate static call graph using NetworkX~\cite{networkx} with the caller-callee relationships generated by Pyan.
    
\begin{figure*}[h]
\centering
\includegraphics[width=\columnwidth]{figures/hla3/hla3_implementation.png}
\caption{Architecture of HCPC tool }~\label{fig:architecture}
\end{figure*}

    \item We generate execution paths from the call graph created in previous step.
Execution paths are grouped using Agglomerative Hierarchical Clustering~(AHC) algorithm provided by Scipy~\cite{scipy} library with \emph{ward} method as distance metric. We have a binary tree structure where leaf nodes are execution paths and other nodes are clusters at different levels. We call these cluster nodes as abstraction nodes. The abstraction nodes have collection of execution paths. For each abstraction node, we generate three properties. For each node, we create node title by applying information retrieval techniques ( Scikit-learn~\cite{scikit-learn} for TFIDF and Gensim~\cite{gensim} for LDA, LSI ) on the method names of all execution paths of a node. Then we produce node summary by summarizing (TextRank by Gensim) method comments of all the execution paths of the node. Last we generate execution patterns by pattern mining among the execution paths of the node~(PrefixSpan \cite{prefixspan}). We write all the node data in a text file named with the project name. Data is written in JSON format where each node is keyed with their ID and they have parent\_id, node title, node summary, execution patterns and execution paths associated with them. 
    \item We have Flask server for interacting with front-end. Client requests which subject system they want to explore and the server returns JSON response with the abstraction tree. 
    \item For the interface of our web application, we have used HTML, CSS, and JQuery. When a specific node is right-clicked, detail information about the node is filled to the node details panel.
    \item We used GoJS for building the abstraction tree diagram. Each abstraction  is a GoJS node and different properties of the abstraction nodes are binded to GoJS nodes. 

\end{enumerate}


\section{Interface}
\label{hla3:interface}
In this section, we will discuss the different components of our HCPC tool shown in Figure \ref{fig:interface}.

\begin{itemize}
    \item \textbf{Abstract Tree Panel(A).} In the panel, the main abstraction tree is presented. The root nodes are presented vertically which can be possible to expand with their child nodes. By right clicking the mouse on a node will load different information of the abstraction node in the right side of the interface.
    \item \textbf{Number of execution paths(B).} As each node in the abstraction tree are a collection of execution paths, we show the number of execution paths for a selected node in this element.
    \item \textbf{Files (C).} In the element, we show the unique files of all the methods of the execution paths belong to.
    \item \textbf{Node summary (D).} In the element, we have provided natural text description of a node. When developers select a node, the text description of the node will appear in the element. 
    \item \textbf{Execution Patterns (E).} In the element, for a selected abstraction node, frequent function call patterns are presented with the file they are associated with. In the current setting, top-10 frequent execution patterns are shown. 
    
    \item \textbf{Execution paths (F).} In the element, we show five execution paths of a selected abstraction node. The execution paths complement the execution patterns by showing a glimpse of the real execution paths. Moreover, when a specific method is searched, the execution paths with the searched method is presented instead first five methods.
    
    \item \textbf{ Node label technique and search panel (G).} The panel has three drop-down boxes. First, developers can select which subject system they want to explore. Second, they can choose which technique to be used for labeling the nodes in abstraction tree. Third, this drop-down box is search enabled and it helps to highlight the nodes which have the searched method in their execution paths.
\end{itemize}

\begin{figure*}[h]
  \centering
  \includegraphics[width=\columnwidth]{figures/hla3/hla3_interface.png}
  \caption{HCPC tool interface }~\label{fig:interface}
\end{figure*}


\section{Guide to use HCPC}
\label{hla3:use_guide}
The tool can be used in two ways. First, a developer new to the code-base can load the abstraction tree which start with top abstraction  nodes. In the node details panel, for each node the number of execution paths, a brief natural text summary, and few frequent execution patterns are presented. Therefore, the developer can start first by observing summary, patterns of the top nodes. Now, the child nodes of the top nodes can be expanded and  similarly explored by observing corresponding node summary and patterns. The developer can continue this way according to their need to get acquainted with the coda-base behavior and high-level concepts in the code-base.

Second, a new contributor to a open source project or someone new to a team can utilize the tool to understand high-level concepts related to a specific method. Developers first start from looking to open issues of a repository to find something work on. The issues are natural text description which provides information regarding a bug or a feature enhancement request. Developers can identify few keywords and use our tool to find matching methods relevant to the keywords. Next, a specific method can be selected to highlight relevant nodes in the tree. The difference between the first approach here is developers will be able to browse the tree with focus to the selected method. The node titles relevant to selected method will be highlighted so that the developer can expand their child nodes. By this way, the developer can start from the high-level concept to low-level source code related concepts for a specific method. By iterating this process, the developer can grasp high-level domain knowledge (with comment summary and IR techniques on function names) alongside insight into program execution scenarios which decreases the overhead due to lack of domain knowledge in the code-base. 

\section{Exploring HCPC for \emph{jupyter\_client} project} 
\label{hla3:case_study}
\textbf{Exploring overview. } We have picked \emph{jupyter-client}\footnote{https://github.com/jupyter/jupyter\_client} as the subject system to show how the tool can be used following the two above mentioned techniques. To discuss the effectiveness of our tool using \emph{jupyter-client}, first we will discuss high level functionalities of \emph{jupyter-client} from their documentation. Later, we will present the information provided by our tool and discuss whether our tool provides similar or more information to comprehend the \emph{jupyter-client} project. \emph{jupyter-client} has three components. First, \emph{kernelspec} deals with specify different type of kernels from predefined files. Second, kernel manager which is responsible for start, stop and signaling kernels for different scenarios. Third, kernel client which is responsible for communicating with kernels for code execution and other tasks \footnote{https://jupyter-client.readthedocs.io/en/stable/index.html}. From the above components we can get an abstract idea of the features of \emph{jupyter-client}. Now, we will discuss the high-level features suggested by HCPC tool. Below we have listed few high-level node summary of the \emph{jupyter-client} project and discuss them with respect to the documentation.

\begin{figure*}[h]
  \centering
  \includegraphics[width=\columnwidth]{figures/hla3/hla3_jupyter_overview.png}
  \caption{HCPC tool overview for jupyter\_client project }~\label{fig:tool_overview_jupyter_client}
\end{figure*}

\begin{itemize}
    \item \emph{ Restarts a kernel with the arguments that were used to launch it. Prepares a kernel for startup in a separate process. Write connection info to JSON dict in self.connection\_file. replace templated args (e.g. Verify realpath is used when formatting connection file). Walks env entries in templated\_env and applies possible substitutions from current env.} 
    
    From this node summary, we can understand \emph{jupyter\_client} restarting kernels, writing connection information to file and creates different kernel environments.
    
    \item \emph{Create a zmq Socket and connect it to the kernel. Start a new kernel, and return its Manager and Client. return zmq Socket connected to the Control channel. Get the stdin channel object for this kernel. Wait for kernel shutdown, then kill process if it doesn't shutdown. Pass a message to the ZMQ socket to send. return zmq Socket connected to the Heartbeat channel. Get the shell channel object for this kernel. Get the iopub channel object for this kernel. Get the control channel object for this kernel. Sends a signal to the process group of the kernel (this. Stops all the running channels for this kernel. return zmq Socket connected to the Shell channel. return zmq Socket connected to the IOPub channel. return zmq Socket connected to the StdIn channel.} 
    
    From this node summary, we observe that the \emph{jupyter\_client} project has ZMQ socket which helps with message communication. It has different channels like iopub, stdin, shell and Heartbeat channel.
    
    \item \emph{load the IPs that point to this machine. populate local and public IPs from flat list of all IPs. return the IP addresses that point to this machine.} 
    
    From this node summary, we can comprehend that the \emph{jupyter\_client} project also deals with public, local IP address of a machine. 

\end{itemize}

From the above text blocks, we can understand that \emph{jupyter-client} is relevant to working with kernels, it uses ZMQ socket to communicate with kernels, and deals with IP addresses of a machine. In addition to the above node summaries when developers see the execution patterns, they can very quickly learn about the domain knowledge of \emph{jupyter\_client} project.

\textbf{Exploring for specific task. } Next, it is possible to browse the tree by focusing on a specific method. In Figure \ref{fig:tool_write_connection_file}, we can see the nodes in the tree are marked to indicate they are relevant to write\_connection\_file method. Developers can investigate the nodes marked to understand relevant concepts of write\_connection\_file method. In Figure \ref{fig:tool_write_connection_file}, at the bottom of the tree we can see execution paths which have write\_connection\_file. At the right side of Figure \ref{fig:tool_write_connection_file}, we can observe node summary and execution patterns for the red marked nodes for better understanding of our target concept. Below we have mentioned and discussed few significant node summary relevant to write in connection file.  

\begin{figure*}[h]
  \centering
  \includegraphics[width=\columnwidth]{figures/hla3/hla3_write_connection_file.png}
  \caption{HCPC tool when focusing on write\_connection\_file method }~\label{fig:tool_write_connection_file}
\end{figure*}

% Discuss write_connection_file
\begin{itemize}
    \item \emph{Create a zmq Socket and connect it to the kernel. return the IP addresses that point to this machine. Write connection info to JSON dict in self.connection\_file. Restarts a kernel with the arguments that were used to launch it. Restarts a kernel with the arguments that were used to launch it. Pass a message to the ZMQ socket to send. Cleanup connection file *if we wrote it*. Given a message or header, return the header. Forgets randomly assigned port numbers and cleans up the connection file. Sends a signal to the process group of the kernel }
    
    From this node summary, we comprehend that in \emph{jupyter\_client} some concepts related to write in connection files are write connection info as JSON dict, cleanup of connection file and forgetting randomly assigned port numbers.
    
    \item \emph{ Restarts a kernel with the arguments that were used to launch it. Prepares a kernel for startup in a separate process. Write connection info to JSON dict in self.connection\_file. replace templated args (e.g. Verify realpath is used when formatting connection file. Walks env entries in templated\_env and applies possible substitutions from current env.}
    
    From this node summary, we comprehend that in \emph{jupyter\_client} some concepts related to write in connection files are restart kernel, creating environments and prepare a kernel startup in separate process.
    
    \item \emph{Load connection info from JSON dict in self.connection\_file. return ip for localhost (almost always 127.0.0.1) set up ssh tunnels, if needed.} 
    
    From this node summary, we comprehend that in \emph{jupyter\_client} some concepts related to write in connection files are set up ssh tunnel, loading connection info from file.
    
    
\end{itemize}

From above discussion with regard to write\_connection\_file method, we can see that HCPC helps to understand relevant concepts for a specific task. 
\section{Conclusion}

In this study, we have proposed two new approaches for improving abstract code summary tree for program comprehension. First, we changed the similarity metrics for comparing execution paths. In previous studies, Jaccard distance is used which only considers the content not the sequence. Therefore, in this study, we have changed the similarity metrics to strike\_a\_match algorithm. Second, we changed the clustering approach for more precise grouping of the execution paths. Previous studies suggested to use the hierarchy tree for browsing. However, from our previous studies we found it is difficult when going to deeper nodes are required as the tree expand step by step. Instead we used cluster flattening technique where all the clusters are merged within a specific distance. We have built an interactive system to explore the new abstraction tree with supporting information alongside searching the tree. To evaluate, the system we have conducted a human subject study. We find that our system can be useful for getting acquainted with a software project and it can also help to accomplish tasks in hand.  


\chapter{Conclusion and Future Plans}
\label{chapter:conclusion}
In this thesis, we have worked on grouping execution paths of a software project for helping developers comprehend the codebase faster and locate related concepts for their tasks in hand. We start with existing works on clustering static execution paths for presenting high-level features in a software project. However, we find some limitations and scope of improvement to make the abstract code summary more usable for the daily activities of software developers. First, we experimented with different information retrieval techniques to find out which techniques provide more helpful label for abstraction nodes. We also proposed using the terms in method names instead of whole method name as input for IR techniques. We also conducted a human subject study to find out how developers rate different IR techniques and compared automatic naming with manual naming by developers. From the study, we found TFIDF with terms in method are much supported by the manual labeling compared to LDA, LSI technique. Moreover, developers preferred words variant than method variant of the labeling technique. Second, we proposed to add additional information such as node summary, execution patterns for each abstraction node to make the abstract code summary tree more comprehensible. We conduct a case study with three different subject systems to find the potential of attaching the two new information for each abstraction node. We found that attaching node summary, and execution patterns can complement node labels for more detailed understanding in relation to source code. However, we observed that using agglomerative cluster tree pose some difficulty to browse as it presents all the clustering step by step. In addition, we noticed that it is difficult to explore the tree targeting some specific keywords. In our third study, we addressed the issue of abstraction tree being overwhelming to browse by simplifying the agglomerative cluster tree using cluster flattening technique. Additionally, we have added abstraction node highlighting technique for browsing the tree targeting specific keywords or methods. To evaluate the usefulness of our technique, we developed a web application called HCPC. We performed a human subject study with an industry project and their developers. From the study, we found that HCPC can help developers get started with a project alongside finding relevant execution patterns for specific tasks in hand.

In future, we plan to adopt automatic method summarizing techniques as in industry settings not every method is properly documented. We also plan to incorporate GitHub project import option for explore in HCPC tool. Moreover, we will add feature to export reports with our analysis result.

%%%%%%%%%%%%%%%%%%%%%%%%%%%%%%%%%%%%%%%%%%%%%%%%%%%%%%%%%%%%%%%%
% The Bibliograpy should go here. BEFORE appendices!
%%%%%%%%%%%%%%%%%%%%%%%%%%%%%%%%%%%%%%%%%%%%%%%%%%%%%%%%%%%%%%%%


% Typeset the Bibliography.  The bibliography style used is "plain".
% Optionally, you can specify the bibliography style to use:
% \uofsbibliography[stylename]{yourbibfile}

\uofsbibliography{reference}

% If you are not using bibtex, comment the line above and uncomment
% the line below.  
%Follow the line below with a thebibliography environmentand bibitems.  
% Note: use of bibtex is usually the preferred method.

%\uofsbibliographynobibtex


%%%%%%%%%%%%%%%%%%%%%%%%%%%%%%%%%%%%%%%%%%%%%%%%%%%%%%%%%%%%%%%%%%%%%%%%%
% APPENDICES
%
% Any chapters appearing after the \appendix command get numbered with
% capital letters starting with appendix 'A'.
% New chapters from here on will be called 'Appendix A', 'Appendix B'
% as opposed to 'Chapter 1', 'Chapter 2', etc.
%%%%%%%%%%%%%%%%%%%%%%%%%%%%%%%%%%%%%%%%%%%%%%%%%%%%%%%%%%%%%%%%%%%%%%%%%%

% Activate thesis appendix mode.
\uofsappendix

% Put appendix chapters in the appendices environment so that they appear correcty
% in the table of contents.  You can use \input's here as well.
\begin{appendices}

\chapter{Simple Calculator program to demonstrate the clustering approach}
\label{appendix:calculator}

\lstinputlisting{appendix/calculator.py}

\end{appendices}

\end{document}
